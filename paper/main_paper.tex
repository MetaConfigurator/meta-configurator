\documentclass[lettersize,journal]{IEEEtran}
\usepackage{amsmath,amsfonts}
\usepackage{algorithmic}
\usepackage{algorithm}
\usepackage{array}
\usepackage{minted}
\usepackage[caption=false,font=normalsize,labelfont=sf,textfont=sf]{subfig}
\usepackage{textcomp}
\usepackage{lipsum}
\usepackage[export]{adjustbox}
\usepackage{stfloats}
\usepackage{url}
\usepackage{verbatim}
\usepackage{graphicx}
\usepackage{cite}
\usepackage{hyperref}
\usepackage{enumitem}
\usepackage{listings}
\usepackage{booktabs}
\hyphenation{op-tical net-works semi-conduc-tor IEEE-Xplore} % todo remove
% updated with editorial comments 8/9/2021
\usepackage{utfsym}
\usepackage{makecell}
\renewcommand{\checkmark}{\usym{1F5F8}}
\newcommand{\cfgfiles}{configuration files}
\newcommand{\cfgfile}{configuration file}
\newcommand{\toolname}{\textit{MetaConfigurator}} % or config assistant


\begin{document}

 \title{Design, Implementation, and Evaluation of a Meta Configuration Tool Using Schema-to-UI Approaches}
 \author{Felix Neubauer, Paul Bredl, Minye Xu, Keyuriben Patel}
%\author{IEEE Publication Technology,~\IEEEmembership{Staff,~IEEE,}
 % <-this % stops a space
%\thanks{This paper was produced by the IEEE Publication Technology Group. They are in Piscataway, NJ.}% <-this % stops a space
%\thanks{Manuscript received April 19, 2021; revised August 16, 2021.}}

% The paper headers
 \markboth{Journal of best software 2023}%
 {Shell \MakeLowercase{\textit{et al.}}: Revolutionizing Software}

%\IEEEpubid{0000--0000/00\$00.00~\copyright~2021 IEEE}
% Remember, if you use this you must call \IEEEpubidadjcol in the second
% column for its text to clear the IEEEpubid mark.

 \maketitle

 \begin{abstract}
  % todo write abstract
 \end{abstract}

 \begin{IEEEkeywords}
  JSON, YAML, configuration, schema, gui
 \end{IEEEkeywords}


 \section{Introduction}\label{sec:introduction} %felix

 Textual formats to structure data, such as JSON, XML and YAML, are human-readable as well as machine-readable.

 For configuration files such formats are often used, since they can be read and maintained by humans, as well as deserialized and used by computer programs.
 The format of those data structures can be defined by so-called schemas, which define the rules the data has to conform to.
 Besides configuration files and the field of software development, other academic fields, such as Biology or Chemistry also use those formats to structure and validate their data.

 To simplify, we will call all file instances using such formats \textit{\cfgfiles}.

 Depending on the content of such \cfgfiles, they can be complex and time-consuming to modify and maintain.
 Tooling, such as graphical user interfaces, can significantly reduce manual efforts and assist the user.
 Those graphical user interfaces, however, require initial effort to be developed, as well as continuous effort in being maintained and updated when the underlying configuration schema changes.
 We intend to tackle this problem by developing a meta program that allows humans to generate such configurator GUIs, which can be shared with others.

 Our approach differs from other schema-to-UI approaches in following:

 \begin{enumerate}
  \item The tool combines the assistance of a GUI with the flexibility and speed of a rich-text editor by providing both in one view
  \item The schema can be defined using the same tool
  \item The defined schemas, the resulting configurator GUI and the modified configuration files can all be shared with others
 \end{enumerate}


 In section~\ref{sec:research} we will introduce related work and introduce existing schema formats as well as schema-to-gui approaches.
 Next we will discuss the design of our solution (section TODO).


% Summary of our approach/idea and reference the different sections, e.g. in Section 1 we will research existing work on that area and then in section 2 blabla.


 \section{Related Work}\label{sec:research}
 % intro: felix

This section covers existing schema languages and existing approaches to generate UIs from them.
% We compare the schema languages briefly and discuss why JSON schema is the most useful language for our approach.
As our research is of a practical nature, we also consider gray literature such as specifications of schemas or websites.
\subsection{Schema Languages}\label{subsec:schemalanguages}

\textit{Schema languages} are formal languages that specify the structure, constraints, and relationships of data, for example in a database or structured data formats.

As this work is concerned with generating a GUI based on a schema, we need to choose a suitable schema language.
The following sections describe existing schema languages.
We will compare them in section~\ref{sec:evaluation-of-schema-languages} to determine which is the most suitable one for this work.

%paul
\subsubsection{JSON schema}

JSON is a common data-interchange format for exchanging data with web services, but also for storing documents in NoSQL databases, such as MongoDB\@\cite{marrs2017json}.
Because of the popularity of JSON, there is also a demand for a schema language for JSON\@.
One such language is JSON schema~\cite{jsonSchema, jsonschemaJSONSchema}.
Listing~\ref{lst:json-schema-example} shows an example of a JSON schema and listing~\ref{lst:json-example} shows an example of a JSON document that conforms to the schema.

JSON schema has evolved to being the de-facto standard schema language for JSON documents~\cite{baazizi2021empirical}.
Schemas for many popular \cfgfile{} types exist.
\textit{JSON schema store}\cite{schemastoreJSONSchema} is a website that provides over 600 JSON schema files for various use cases.
The supported file types include for example Docker compose or OpenAPI files.
%\cite{barbaglia, ChaeronySiffa2022} give further examples of JSON schema used in practice.

We remark that JSON schema and other schema languages for JSON can also be applied to YAML as JSON and YAML documents are of a similar structure (JSON is a subset of YAML).
Some syntactical details of YAML can, however, not be expressed with JSON schema.

% minye
\begin{lstlisting}[language=json,firstnumber=1,caption={JSON schema example},captionpos=b,label={lst:json-schema-example}]
{
  "$id": "https://example.com
  /person.schema.json",
  "$schema": "https://json-schema.org
  /draft/2020-12/schema",
  "title": "Person",
  "type": "object",
  "properties": {
    "firstName": {
      "type": "string",
      "description": "first name."
    },
    "lastName": {
      "type": "string",
      "description": "last name."
    },
    "age": {
      "description": "Age",
      "type": "integer",
      "minimum": 0
    }
  }
}
\end{lstlisting}


\begin{lstlisting}[language=json,firstnumber=1,caption={JSON example for the schema in listing}~\ref{lst:json-schema-example},captionpos=b,label={lst:json-example}]
{
  "firstName": "John",
  "lastName": "Doe",
  "age": 21
}
\end{lstlisting}

% felix
\subsubsection{XSD and DTD}
For XML the two de-facto standard schema languages are Document Type Definition (DTD)\cite{dtd_spec} and XML Schema Definition (XSD)\cite{xsd_spec}.
XSD is the newer and more expressive format and in large parts replaces and supersedes the more limited format DTD\cite{dtd_vs_xsd}.
It is recommended by W3C as a schema language for XML documents\cite{xsd_spec}.
Multiple other schema languages have been proposed and developed but are relatively unknown compared to XSD\cite{xml_schemas_1,xml_schemas_2}.

\subsubsection{Other schema languages} %Minye
We also consider the following schema languages:
\begin{enumerate}[label=(\alph*)]
    \item CUE (Configure, Unify, Execute)\cite{cuelang} is a data validation and configuration language, which can be used with various data formats, such as JSON and YAML (it is a superset of both).
    It has several use cases, especially in configuration and data validation.
    \item Apache Avro~\cite{Apache-Avro} is an open-source project that provides data serialization and data exchange services for Apache Hadoop.
    It uses a JSON-based schema language.
    \item JSON Type Definition (JTD)~\cite{rfc8927} is a schema language for JSON documents, which is significantly simpler than JSON schema.
    \item Type Schema~\cite{Kappestein_2023} is a schema language for JSON documents, similar to JSON Type Definition but using a different syntax.
    \item GraphQL schema language~\cite{graphQL} is a schema language for GraphQL APIs.
    \item Protocol Buffers~\cite{protobufProtocolBuffers} is a language for data serialization by Google.
\end{enumerate}

We do not consider any graphical modeling languages, such as UML or ER diagrams, as they are not text-based.
Although they can be converted to text-based formats, their main purpose is to model data structures and relationships between them.
We also do not consider any ontology languages, such as OWL or RDF Schema, as they are not intended for data validation but rather for knowledge representation.
Future work could investigate if such languages are also useful for our use case.
Finally, we do not consider any programming languages as schema languages.
Technically, programming languages can be used to define data structures and constraints, but they are not intended for this purpose, and it would be very challenging to generate a GUI from them.

\subsection{Existing Approaches}\label{subsec:existing-approaches}
% felix
Our work focuses on assisting users in creating and maintaining \cfgfiles{} so that they are valid and adhere to a predefined schema.

%There exist several approaches that attempt to make maintaining \cfgfiles easier for the user.
%Most of these approaches depend on the schema of the configuration file to be known.
There exist techniques to validate \cfgfiles{} against a schema~\cite{JSON_schema_vailidation,JSONValidation,baeldung_2023}.
Usually, schema validation is done only internally, e.g., by web services or libraries.
However, there exist also approaches that use the schema to assist the user in creating and maintaining \cfgfiles.
IDEs, such as Visual Studio Code or IntelliJ IDEA, can validate \cfgfiles{} against a schema and provide the user with error messages.
Those IDEs also provide other features, such as auto-completion, syntax highlighting, and tooltips.
However, they typically do not provide a graphical user interface (GUI) for editing the \cfgfiles{} based on the schema.


\subsubsection{Form generation}\label{subsubsec:schema-to-gui}
% minye and felix and paul

Related to our work are approaches that generate a GUI from a schema.
This section covers form generators, i.e., approaches that generate a web form from a schema.
Such forms can assist the user in a multitude of ways, such as by tooltips, auto-completion (Figure~\ref{fig:gui_advantage_autocomplete}) and dropdown menus (Figure~\ref{fig:gui_advantage_choiceselection}).
By inherently adhering to the schema structure (in most cases), editing data with such GUIs significantly reduces configuration mistakes caused by the user.
Users who are not very familiar with the configuration schema profit most from the GUI assistance, but even experienced users benefit from it.

\begin{figure}[htb]
    \centering
    \includegraphics[width=2.5in]{figures/gui_advantage_autocomplete}
    \caption{Auto-Completion}
    \label{fig:gui_advantage_autocomplete}
\end{figure}

\begin{figure}[htb]
    \centering
    \includegraphics[width=2.5in]{figures/gui_advantage_choiceselection}
    \caption{Choice Selection}
    \label{fig:gui_advantage_choiceselection}
\end{figure}

%paul
There exist various approaches that generate web forms from a schema, for different frontend frameworks, e.g.,
\textit{React JSON Schema Form}\cite{githubGitHubRjsfteamreactjsonschemaform},
\textit{Angular Schema Form}\cite{githubGitHubJsonschemaformangularschemaform},
\textit{Vue Form Generator}\cite{githubGitHubVuegeneratorsvueformgenerator},
\textit{JSON Forms}\cite{jsonformsMoreForms},
\textit{JSON Editor}\cite{jsoneditoronlineJSONEditor}, and
\textit{JSON Form}\cite{githubGitHubJsonformjsonform}.
Those approaches are all based on JSON schema and generate a form that can be filled out by the user and
the resulting JSON document is validated against the schema.
If the user enters invalid data, the form shows an error message.
The generated forms usually have a specific component for each type of data, e.g.\ a text field for strings or a number field for numbers,
similar to our approach.
Figure~\ref{fig:jsonforms} shows an example of a generated form using JSON Forms.

Those techniques, however, only provide the GUI for editing the data, but not a text-based editor.
A text-based editor is useful, especially for experienced users, who prefer to edit the data directly.
Also, these techniques do not provide a way to edit the schema itself, but only the data.
The most significant limitation of all except the last two of the given approaches is that they also require a ``UI schema'' in addition to the JSON schema, which is used to configure the generated form.
While these configurations can be used to customize the generated form, they also need to be created and maintained by the schema author.
Consequently, those approaches cannot be used to generate a GUI for any arbitrary schema, but manual effort is required to create the UI schema.

\begin{figure}[htb]
    \centering
    \includegraphics[width=2.5in]{figures/jsonforms}
    \caption{JSON Forms, example for a generated form}
    \label{fig:jsonforms}
\end{figure}

Adamant~\cite{siffa2022adamant} is a JSON-schema based form generator specifically designed for scientific data.
It generates a GUI from a JSON schema, allows editing and creating JSON schema documents
and differentiates between a schema edit mode and a data edit mode.
It supports a subset of JSON schema, which is sufficient for many use cases.
In addition to that, it supports the extraction of units from the description of a field, which is helpful for scientific data.
Figure~\ref{fig:adamant} shows an example in the schema edit mode.
%Adamant differs to \toolname{} in that it does not provide a text-based editor for the schema and that it is specifically designed for scientific data,
%while our approach is more general.
Limitations of Adamant are first, it does only support a subset of JSON schema, which is sufficient for many use cases, but not for any arbitrary schema.
Second, it does not provide a text-based editor for neither the schema nor the data.
Finally, it is specifically designed for scientific data, which makes it less suitable for other use cases, especially large and complex schemas.

\begin{figure}[htb]
    \centering
    \includegraphics[width=2.5in]{figures/adamant}
    \caption{Adamant, example for a form in edit mode}
    \label{fig:adamant}
\end{figure}


\subsubsection{Schema editors}\label{subsubsec:schema-editors}
In \toolname{} we aim to provide a GUI for both editing \cfgfiles{} and editing the schema.
For the latter, there exist several so-called schema editors, which are tools for creating and editing schemas that are
either text-based or graphical (or both).

\textit{JSON Editor Online}\cite{jsoneditoronlineJSONEditor} is a web-based editor for JSON schemas and JSON documents.
It divides the editor into two parts, where one part can be used to edit the schema and the other part can be used to edit a JSON document,
which is validated against the schema.
The editor provides various features, such as syntax highlighting and highlighting of validation errors (Figure~\ref{fig:jsoneditoronline}).
It provides a text-based or tree-based view for editing the JSON documents.
For simple objects that are not further nested, it also provides a table-based view (Figure~\ref{fig:jsoneditoronline_table}).
However, the features of the editor are very limited.
For example, it does not provide any assistance for the user, such as tooltips or auto-completion.
For new documents, it does not show the properties of the schema, so the user has to know the schema beforehand.

There also exists a variety of schema editors that are paid software, such as \textit{Altova XMLSpy}\cite{altovaEditorXMLSpy},
\textit{Liquid Studio}\cite{liquidtechnologiesJSONSchema}, \textit{XML ValidatorBuddy}\cite{xmlbuddyEditorValidator},
\textit{JSONBuddy}\cite{jsonbuddyJSONSchema}, \textit{XMLBlueprint}\cite{xmlblueprintEditorXMLBlueprint},
and \textit{Oxygen XML Editor}\cite{oxygenxmlCompleteSolution}.
Those are editors for XML or JSON schema, mostly with a combination of text-based and graphical views.
These tools are not web-based and not open-source.
Furthermore, they do not focus on editing a JSON document based on a schema,
but rather only on editing the schema itself.
\begin{figure}[htb]
    \centering
    \includegraphics[width=3in]{figures/jsoneditoronline}
    \caption{JSON Editor Online}
    \label{fig:jsoneditoronline}
\end{figure}
\begin{figure}[htb]
    \centering
    \includegraphics[width=3in]{figures/jsoneditoronline-table}
    \caption{JSON Editor Online, table view}
    \label{fig:jsoneditoronline_table}
\end{figure}


\subsubsection{Schema visualization}\label{subsubsec:schema-visualization}
Generating a GUI from a schema is related to schema visualization, for which several techniques exist~\cite{Frasincar2006, SILVA201928, 10.1145/1317353.1317362, 1173142}.
However, the focus of schema visualization is on providing a static visual representation of the schema
and not on providing a GUI for editing the schema.
Thus, we do not consider schema visualization approaches in this work.
However, future work can investigate how such techniques could be embedded in our approach.


\section{Evaluation of Schema Languages}\label{sec:evaluation-of-schema-languages}


We evaluate the schema languages mentioned in section~\ref{subsec:schemalanguages} to determine which is the most suitable one for this work.


\subsection{Evaluation Criteria}\label{subsec:evaluation-criteria} % evaluation by paul, some criteria ideas by felix

Ideally, the schema language of \toolname{} is both popular and supported by numerous tools and libraries as well as expressive enough to express the features we need.
We use the following criteria and metrics:
\begin{enumerate}
% metric: Stackoverflow question # with schema language as tag
    \item \textbf{Practical usage} --- Ideally our approach uses a schema language that is already known by many developers.
    As indicator of the practical usage we use the approximate search results on stackoverflow.com as metric.
    We acquire the results by querying the google search engine with the name of the schema language and ``site:stackoverflow.com'', which limits the search results to stackoverflow.com.
    This metric might also correlate with the complexity of the schema language as a more complex to use schema language will likely lead to more questions asked on the site.
    Nevertheless, we assume that a significantly higher number of results indicates that a language is more known than others.

    Additionally, we investigate how well the schema languages are supported by IDEs and code libraries:
    \begin{enumerate}
        \item \textit{Tool support} --- We used the 10 most popular IDEs\cite{mostpopularides} and checked if the IDE supports the schema language either natively or by a plugin.
        Here, support means that either the IDE is capable of validating documents against a schema in the schema language or supports creating schema files, e.g., by using syntax highlighting for the schema language.
        % # node modules with schema language keyword....
        \item \textit{Library support} --- As we implement a web-based tool, JavaScript or TypeScript based tools are helpful for our approach, e.g., so we can reuse a package for schema validation.
        We investigate the number of node modules existing that are related to the schema languages by querying the node module search on \url{www.npmjs.com} with the name of the schema language.

    \end{enumerate}

    % # of cases fulfilled from below
    \item \textbf{Expressiveness} --- We evaluate how expressive each of the schema languages is, i.e., what possible constructs the language is able to express.
    We define eight requirements on the language features that we consider helpful for our approach.
    The number of requirements a schema language fulfills is our metric that indicates how expressive the language is.
    Table~\ref{tab:comparison} reports the results.
    The nine requirements are:
    \begin{enumerate}
        \item \textit{Simple types} --- This is fulfilled if the schema language provides the possibility to define simple data types, at least strings, numeric types, and a boolean type.
        This is a fundamental feature for our approach.
        \item \textit{Complex types} --- This is fulfilled if the schema language provides the possibility to define complex data types, at least records and arrays.
        This is crucial feature for our approach as configuration files are often structured data rather than plain key-value pairs.
        \item \textit{Descriptions} --- This is fulfilled if the schema language provides the possibility to add descriptions to fields.
        This is helpful in a schema-to-GUI approach as the description can be shown to the user, providing potential helpful information on how a field should be filled.
        \item \textit{Examples} --- This is fulfilled if the schema language provides the possibility to add example values.
        This is helpful in our approach as the example values can serve as placeholders in the GUI editor.
        \item \textit{Default values} --- This is fulfilled if the schema language provides the possibility to add default values which are assumed in an absence of a value.
        This helpful information can be displayed to the user or used as placeholder values.
        \item \textit{Optional values} --- This is fulfilled if the schema language provides the possibility to declare values as optional or required.
        Often it is not necessary to provide all values in a configuration file, so it is helpful to mark fields as required or optional in the GUI editor.
        \item \textit{Constraints} --- This is fulfilled if the schema language provides the possibility to constrain values of fields, e.g., maximum length of strings.
        To be exact, for this evaluation we require that at least two of the following constraints can be expressed by the schema language:
        \begin{itemize}
            \item The length of strings can be limited.
            \item The range of numeric types can be limited, e.g., to only positive values.
            \item The valid values of a field can be restricted to a finite amount of values (enumeration).
            \item The format of a string field can be constrained to a certain pattern.
        \end{itemize}
        This is a helpful feature for our approach as often not all possible values are valid for specific fields in configuration files.
        \item \textit{Conditions} --- This is fulfilled if the schema language provides the possibility to define conditional dependencies between fields.
        This is an advanced feature that is helpful because it allows to express that a particular field must be given only if another field has a specific value.
        \item \textit{References} --- This is fulfilled if the schema language provides the possibility to define reusable sub-schemas that can be referenced in other parts of the schema.
        This is often useful in practice to reuse common data structures.

    \end{enumerate}

    %\item \textbf{Support for XML, YAML, or JSON} ---

    % For our approach we aim to use the same or at least a very similar editor for both editing the actual \cfgfiles and the schema files. Consequently, the schema language should be a subset of the language that the \cfgfiles are written in, i.e. the document language. For example JSON schema files are JSON files and JSON schema is used to validate JSON files, so here this criteria is fulfilled. In constrast, DTD is a schema language for validating XML files but a DTD file is not a valid XML file.

\end{enumerate}


\begin{table*}[]
    \centering
    \caption{Evaluation of different schema languages\label{tab:all}}
    \begin{tabular}{@{}lrrrr@{}}
        \toprule
        \textbf{Schema language} &
        \textbf{\# Search results } &
        \textbf{IDE support} &
        \textbf{\# Node packages} &
        \textbf{Expressiveness} \\ \midrule
        JSON schema &
        245,000 &
        8 / 10 &
        4,536 & 9 /9 \\
        XSD & 151,000 & 8 / 10 & 116 & 8 / 9 \\
        DTD & 69,700 & 9 / 10 & 34 & 6 / 9 \\
        CUE & 10,500 & 4 / 10 & 97 &   8 / 9 \\
        Avro & 20,000 & 8 / 10 & 211 &  5 / 9 \\
        JSON Type Definition (JTD) &  109 & 0 / 10 &  17 & 5 / 9 \\
        TypeSchema &  8,450 & 0 / 10 & 5 & 8 / 9 \\
        Protobuf &  44,800 & 9 / 10 & 1,210 & 4 / 9 \\
        GraphQl schema & 31,000 & 7 / 10 & 1,509 & 6 / 9\\ \bottomrule
    \end{tabular}
\end{table*}
%\label{fig:my_label}
%\end{figure}

% IDE                | JSON schema | XSD | DTD | CUE | Avro | protobuf | GraphQL schema |
% Visual Studio      | yes         | yes | yes | x   | yes  | yes      | yes            |
% Visual Studio Code | yes         | yes | yes | yes | yes  | yes      | yes            |
% Eclipse            | yes         | yes | yes | x   | yes  | yes      | x              |
% pyCharm            | yes         | yes | yes | yes | yes  | yes      | yes            |
% Android Studio     | yes         | yes | yes | yes | yes  | yes      | yes            |
% IntelliJ           | yes         | yes | yes | yes | yes  | yes      | yes            |
% NetBeans           | x           | yes | yes | x   | x    | yes      | x              |
% RStudio            | x           | x   | x   | x   | x    | x        | x              |
% Atom               | yes         | x   | yes | x   | yes  | yes      | yes            |
% Sumblime Text      | yes         | yes | yes | x   | yes  | yes      | yes            |
% ======================================================================================|
%                    | 8 / 10      | 8   | 9   | 4   | 8    | 9        | 7              |

\begin{table*}
    \centering
    \caption{Comparison of expressiveness of different schema languages
    \label{tab:comparison}}
    \begin{tabular}{@{}llllllllllr@{}}
        \toprule
        \textbf{Schema language} &
        \thead{Simple \\ types} &
        \thead{Complex \\ types} &
        \thead{Descriptions} &
        \thead{Example \\ values} &
        \thead{Default \\ values} &
        \thead{Optional \\ values} &
        \thead{Constraints} &
        \thead{Conditions}  &
        \thead{References}  &
        \thead{Result}\\ \midrule
        JSON schema &
        \checkmark &
        \checkmark &
        \checkmark &
        \checkmark &
        \checkmark &
        \checkmark &
        \checkmark &
        \checkmark &
        \checkmark & 9 / 9\\
        XSD &
        \checkmark &
        \checkmark &
        \checkmark &
        x &
        \checkmark &
        \checkmark &
        \checkmark &
        \checkmark &
        \checkmark & 8 / 9 \\
        DTD &
        \checkmark &
        \checkmark &
        x &
        x &
        \checkmark &
        \checkmark &
        x &
        \checkmark &
        \checkmark & 6 / 9\\
        CUE &
        \checkmark &
        \checkmark &
        \checkmark &
        \checkmark &
        x &
        \checkmark &
        \checkmark &
        \checkmark &
        \checkmark & 8 / 9\\
        Avro &
        \checkmark &
        \checkmark &
        x &
        x &
        \checkmark &
        \checkmark &
        x &
        x &
        \checkmark & 5 / 9\\
        JTD &
        \checkmark &
        \checkmark &
        x &
        x &
        x &
        \checkmark &
        x &
        \checkmark &
        \checkmark & 5 / 9\\
        TypeSchema &
        \checkmark &
        \checkmark &
        \checkmark &
        x &
        \checkmark &
        \checkmark &
        \checkmark &
        \checkmark &
        \checkmark & 8 / 9\\
        protobuf &
        \checkmark &
        \checkmark &
        x &
        x &
        x &
        \checkmark &
        x &
        x &
        \checkmark & 4 / 9 \\
        GraphQL schema &
        \checkmark &
        \checkmark &
        \checkmark &
        x &
        \checkmark &
        \checkmark &
        x &
        x &
        \checkmark & 6 / 9\\
        \bottomrule
    \end{tabular}
\end{table*}

% paul
\subsection{Evaluation results}\label{subsec:evaluation-results}

Tables~\ref{tab:all} and~\ref{tab:comparison} show the results of our evaluation.
We come to the conclusion that JSON schema is sufficiently popular and expressive that we choose to use it as the schema language for our approach.
The other schema languages are either less expressive or less popular.
This result is in line with the work of Baazizi et al.~\cite{baazizi2021empirical}, who also found over 80,000 JSON schema files on GitHub,
and with their claim that JSON schema is the de-facto standard for JSON schema languages.

% paul

%\subsection{Schema inference}\label{subsec:schema-inference}
%
%Schema inference is the process of deriving a schema from existing data.
%For our use case, this means inferring JSON schema from JSON documents.
%Frozza et al.~\cite{8424731} and Klettke et al.~\cite{klettke} present algorithms for JSON schema inference from JSON data of NoSQL data storages.
%Baazizi et al.\cite{Baazizi2019} also investigate schema inference from massive data sets but their approach uses its own type system rather than JSON schema.
%
%In our tool we only aim to infer a schema from a single sample, as an optional assisting feature for our users, for which various libraries exist~\cite{githubGitHubJsonsystemspublic, githubGitHubSaasquatchjsonschemainferrer, probst_siegel_2023}.



 \section{Design}\label{sec:design}
 % TODO: explain why and how we define our own meta schema based on json schema, because we use just subset of its features and because our schema is used to generate schema editor GUI and therefore should only show what is supported. Also we want to add descriptions and maybe more. Also limit options to not overwhelm and confuse the user. 


% felix
\subsection{Functionality}\label{subsec:functionality} % todo maybe rename it requirements?
Before we dive into the architecture and detailed design of the tool, this section sketches what functionality the tool should have, from the viewpoint of a user.

Figure \ref{mockup_gui_config} shows how the tool could look like for the user.


\begin{figure*}[!t]
    \includegraphics[width=\textwidth]{figures/mockup_gui_config}
    \caption{Sketch of the Tool. Workflow: the user edits their use-case specific config file based on their use-case specific schema}
    \label{mockup_gui_config}
\end{figure*}

% TODO: add numbered annotations to the screenshot and explain them here
\begin{enumerate}
    \item Blabla this is the menu bar
\end{enumerate}


Besides the workflow of editing a config file, the user can also use the tool to create a new use-case specific schema.
This is illustrated in figure \ref{mockup_gui_schema}.
The schema called \textit{MyPersonSchema} that was used in figure~\ref{mockup_gui_config} is defined in figure~\ref{mockup_gui_schema}.
As a schema is a \cfgfile{} itself, it can be treated as such and the tool can offer assistance accordingly.
Note that whenever the user edits a \cfgfile{} using the tool, they do so using some underlying schema.
When editing a schema file, the underlying schema will be the schema of JSON Schema.

\begin{figure*}[!t]
    \includegraphics[width=\textwidth]{figures/mockup_gui_schema}
    \caption{Sketch of the Tool. Workflow: the user creates a schema for their use-case, based on JSON Schema}
    \label{mockup_gui_schema}
\end{figure*}

% TODO: add numbered annotations to the screenshot and explain them here
\begin{enumerate}
    \item Blabla this is the menu bar
\end{enumerate}


% felix

\subsection{Architecture}\label{subsec:architecture} % todo diagram illustrating the architecture
The core of our tool is a single source of truth data store that contains the current user configuration data (as a JavaScript Object).
With this data store we can bidirectionally connect what we call ``editor panels''.
An editor panel is a modular component that the user of the tool can access to modify the config data in an indirect way.
It might be implemented as a raw text editor, a graphical user interface or any other way in which the data can be presented to the user.
All editor panels are independent and do only have access to the data store but not to each other.
Every editor panel subscribes to the changes of the data store, so it can be updated accordingly whenever the data in the store is changed.
Additionally, every panel has the capabilities of updating the data store themselves, which is done when the user modifies the data in the editor panel.
The following example use-cases illustrate the capabilities of this architecture:

\begin{itemize}
    \item Format converter: one panel shows the data in a rich-text editor in JSON format, a second panel shows the data in a rich-text editor in YAML format. Any semantic data change on one panel will cause the same semantic change in the other panel.
    \item Split-Screen Editor: one panel shows the data in a rich-text editor, a second panel shows the data in a GUI. This way the user can have the efficiency of a text editor, but also the assistance of a GUI at the same time. Any semantic data change on one panel will be forwarded to the other panel. The GUI editor panel would require some data schema.
    \item The Split-Screen Editor could be implemented for different data formats, such as YAML, JSON and XML. The architecture allows any data format as long as there exists a mapping from this data format to a JavaScript Object and back.
\end{itemize}

% TODO: Add diagram with store in center in several panels with bidirectional connection of subscribe/update

% felix

\subsubsection{Single Source of Truth Data Store}
this is the core of the tool.
The panels can subscribe to this store to receive updates whenever data is changed.
Also, panels can trigger changes of the data in the store.
Besides the current configuration data, the store also stores the \textit{path of the currently selected data entry} and the schema that is currently being used.

% felix

\subsubsection{Text Editor Panel} % todo instead of writing in future, write in simple present
For the text editor panels, we embed a rich-text editor that already supports syntax highlighting and other useful features.
We add validation of whether the text is well-formed according to the JSON/YAML/XML Standard and schema validation. % todo it already supports that
The architecture allows for having one text editor panel that supports multiple languages, as well as for having separate text editor panels, one for each language.
The panels subscribe to the data store.
Whenever the configuration data is changed in the store, the panels will take the new configuration data JavaScript Object, serialize it into the given language and replace the text in the text editor with the new serialized data.
The action of replacing the text in the text editor will cause formatting and comments to be lost.
An alternative to replacing the complete text in the text editor, whenever data in the store is changed, would be to only replace the section of the text, which corresponds to the change.
This would require on the one hand identifying which part of the data is affected by the change and on the other hand understanding of the data within the serialized text in a way that it can be manipulated (e.g.\ navigating within the hierarchical data structure and changing values of a given path). % todo state that this is out of scope and may be part of future work
However, even such deep understanding of the text would wipe out non-default formatting at the sections affected by change.
We tackle those difficulties in the following way: we accept that user-specific text formatting might be undone by our tool.
To allow for different styles of formatting, we will provide the user with global formatting style settings (such as level of indentation or whether in YAML strings should be in quotation marks or not). % todo actually implement this
Whenever the configuration data JavaScript Object is serialized into text, we apply those formatting style settings.
Second, to deal with the loss of comments, we implement a technique that keeps track of any comments in the text and then restores them after the text is replaced. % todo either write that we don't support it for now or implement it

When the user edits the text in the text editor, the text is deserialized into a JavaScript Object and sent to the data store, which then updates the configuration data object and notifies all other subscribed panels of the change.

% felix

\subsubsection{GUI Assistance Panel}
The GUI assistance panel(s) directly work with the given schema and provide the user with corresponding GUI elements, such as a checkbox for a boolean data structure or a text field for a string data structure.
Additional GUI elements, such as tooltips (showing the description of a data field) are used to support the easier.
The GUI elements are constructed in the following manner: a schema is seen as a hierarchical tree of data field definitions and their corresponding constraints.
A data field can either be simple (string, boolean, number, \ldots) or complex (array or dictionary of data fields).
Every schema has a root data field.
The GUI element for this root data field is constructed. % todo describe tree generation
When constructing the GUI element for a complex data field, this includes constructing the GUI elements for all child data fields too.
This way, the whole schema tree is traversed and GUI elements for all entries are constructed.
To avoid overwhelming the user with too many GUI elements, the ones with child elements can be expanded or collapsed by the user and only a limited amount of them is expanded by default.
By design, each of these constructed GUI elements is mapped to their corresponding data field (in other words: to a path in the data structure).
The initial values of all GUI elements are taken from the data in the store, by accessing the data at the given paths.
Whenever the values in a GUI element are adjusted by the user, the data in the store will be updated with the new values.

In the following, the corresponding GUI elements for the different JSON Schema data types and constraints are shown:

\textbf{Boolean}


\textbf{Boolean}

% todo: describe how it works. Then also describe for all the different data types, how we intend to implement the GUI aspect. E.g. for boolean we have checkbox


 \section{Implementation}\label{sec:implementation}
 We use vue.js (TODO citation) as the UI framework for our tool, combined with PrimeVue (TODO citation) as the UI component library and Tailwind CSS (TODO citation) for CSS utility.
A detailed list of all libraries used can be found in the wiki of our GitHub repository (TODO link).
Table~\ref{tab:libraries} shows an overview of the most important libraries used and their purpose.

% todo insert most important libraries used and their purpose
\begin{table}[!t]
    \caption{Libraries used in the implementation of our tool}
    \label{tab:libraries}
    \centering
    \begin{tabular}{ll}
        \toprule
        \textbf{Library} & \textbf{Purpose}     \\
        \midrule
        vue.js           & UI framework         \\
        PrimeVue         & UI component library \\
        Tailwind CSS     & CSS utility          \\
        \bottomrule
    \end{tabular}
\end{table}

\subsection{Code Editor}\label{subsec:code-editor}

The code editor is a text editor program designed for editing the \cfgfiles.
In our user case, we use the popular editor \textbf{Ace Editor}. \cite{Ace-Editor}
It is a browser based editor that matches and extends the features, usability and performance of existing native editors such as TextMate, Vim or Eclipse.
It can be easily embedded in any web page or JavaScript application.

\subsubsection{Features}

To make our code editor more user-friendly, we implemented several features, which are described in the following.

\paragraph{Schema Validation}
To ensure that the data is valid as soon as it is entered by the user, we use schema validation.
Therefore, the config data get validated based on the schema that is currently loaded.
We choose to use \textbf{Ajv JSON schema validator} \cite{ajv-validator}.
It supports the newest JSON schema draft 2020-12.
Ajv firstly generates the input JSON Schemas into super-fast validation functions.
If the config data is not valid according to the schema, there will be a red error hint on left side of the code editor.
In our case, we have a helper class for the validation service, so that it's more convenient to deal with validation.

\paragraph{Support for JSON and YAML}
There is a lot build-in support for different languages in the ace editor.
In our project, we mainly support JSON and YAML. It could be switched in the Settings page.

\paragraph{Concrete Syntax Tree}
% TODO: how we get cursor position and path correctly using cst

\paragraph{Syntax Highlighting}
We have also implemented syntax highlighting for both JSON and YAML.
With this functionality, users could better recognize different components in the data, like properties or values.

\paragraph{Editor Operations}
There are many other useful functionalities we have implemented in the code editor.
For example, the undo and redo operations.
Dragging and dropping files in the code editor.
Font Size configuration in the Settings.


\subsection{GUI Editor}\label{subsec:gui-editor}

The GUI editor is a component that allows the user to edit the configuration data in a GUI, which is generated based on the schema of the configuration data.
It is structured in a table-like way, where each row represents a key-value pair of the configuration data.
Arrays elements are represented similarly, where the index of the array element is the key and the value is the array element itself.
Figure~\ref{fig:gui-editor} shows the GUI editor component with an example schema and configuration data.

\begin{figure}[!t]
    \centering
    \includegraphics[width=\columnwidth]{figures/gui-editor} % todo replace with screenshot
    \caption{GUI Editor Component}
    \label{fig:gui-editor}
\end{figure}

To allow this representation of the schema, we do some preprocessing of the \textbf{schema}, which is described in section~\ref{subsec:schema-preprocessing}.

\subsubsection{Features}\label{subsubsec:gui-editor-features}

To assist the user in editing the configuration data, the GUI editor offers a set of features, which are described in the following.

\paragraph{Traversal of the Data Tree}
By default, only the first level of the data tree is shown.
The user can expand the data tree by clicking on the arrow next to the key of an object or array.
This will show the sub-properties of the object or the elements of the array.
We limit the depth of the data tree to a configurable value, to prevent the GUI editor becoming too overwhelming.
However, the user can also click on the property name or array index to \textit{zoom in} to that element.
This will show the sub-properties of that element at the top level, as if that property was the root of the data tree.
The breadcrumb at the top allows the user to see which path the GUI editor currently shows and to navigate back to upper levels of the tree.

% todo add figure to illustrate

\paragraph{Type specific components}

% todo write

\paragraph{Remove Data}
The user can delete properties or array elements from the data by clicking on the $\times$ button next to the edit field.
This button is only shown if the property is not required.

\paragraph{Schema Information Tooltip}
When the user hovers over the property key or array index, an overlay is displayed which contains all information from the schema about that property.
We manually implemented a generation of a textual description for each of the JSON schema keywords.
This feature helps the user to understand the constraints and the meaning of a property.

\paragraph{Highlighting Schema Validation Errors}
When the configuration data does not comply to the schema, the corresponding elements are underlined in red.
This way, the user knows where any errors are.
Additionally, when hovering over the property name, more details about the error are shown.
% todo add screenshot

\subsection{Schema preprocessing}\label{subsec:schema-preprocessing}

To represent the schema in the GUI editor, we preprocess the schema.
We differentiate between three ways of preprocessing:
A one-time preprocessing step when loading the schema, an internal preprocessing that happens at every layer of the schema tree,
and calculating an effective schema that happens everytime the configuration data changes.

\subsubsection{One-time Preprocessing Step}
When the schema is loaded, we perform a one-time preprocessing step that currently only involves migrating the schema to the newest version,
as described in section~\ref{subsec:json-schema-versions}.
The user will be informed about this step and also prompted with a dialog, when the schema file does not define which JSON schema version it uses.
After the migration, the resulting schema is loaded into the tool in the \textit{Schema Editor} page.

% todo maybe describe how bundling might make sense here

\subsubsection{Internal preprocessing}
This preprocessing steps are mainly used to generate the GUI editor and thus are internal steps that will not be visible for the user.
They happen at every layer of the schema tree lazily, only when required.
This lazy preprocessing is required as schemas can have circular references, which would lead to infinite loops.
In the following, we describe the preprocessing steps in details.

\paragraph{Resolving references}
JSON schema uses the \texttt{\$ref} keyword to reference other schemas.
This can either be references to schemas in the same file (using the \texttt{\$defs} keyword), references to other local files,
or references to schemas at a URL in the web.
We currently only support references to schemas in the same file.
These are lazily resolved as the first preprocessing step.
Listing~\ref{listing:preprocessing-example} shows an example schema, Listing~\ref{listing:reference-resolving} shows the equivalent example after
this first preprocessing step.

\begin{listing}[!h]
    \begin{minted}[frame=single,
        framesep=3mm,
        linenos=true,
        xleftmargin=15pt,
        tabsize=4]{js}
{
  "title": "NonEmptyString",
  "$ref": "#/$defs/nonEmptyString",
  "$defs": {
    "nonEmptyString": {
      "type": "string",
      "minLength": 1
     }
  }
}
    \end{minted}
    \caption{Simple JSON schema before reference resolving}
    \label{listing:preprocessing-example}
\end{listing}

\begin{listing}[!h]
    \begin{minted}[frame=single,
        framesep=3mm,
        linenos=true,
        xleftmargin=15pt,
        tabsize=4]{js}
{
  "allOf": [
   {
     "title": "NonEmptyString"
   },
   {
     "type": "string",
     "minLength": 1
   }
  ],
  "$defs": {
    "nonEmptyString": {
      "type": "string",
      "minLength": 1
     }
  }
}
    \end{minted}
    \caption{Simple JSON schema after reference resolving}
    \label{listing:reference-resolving}
\end{listing}

\paragraph{Resolving allOfs}

The \texttt{allOf} keyword in JSON schema specifies that all of the schemas in the given array must be valid.
To simplify any other operation on the schema, we aim to merge the schemas in the allOf array to one equivalent schema.
As the first step, we do a recursive step by preprocessing all the schemas of the allOf array.
Then, we use the \textit{mergeAllOfs} library % todo citation
for this task.
Listing~\ref{listing:resolved-allOf} shows the previous example schema after this step.
It is important to note that this library only supports a few keywords of JSON schema, most notably the
\texttt{properties} and \texttt{items} keyword.
Hence, the support for allOf and any other keywords for which we use this in the preprocessing is limited.

\begin{listing}[!h]
    \begin{minted}[frame=single,
        framesep=3mm,
        linenos=true,
        xleftmargin=15pt,
        tabsize=4]{js}
{
  "title": "NonEmptyString"
  "type": "string",
  "minLength": 1,
  "$defs": {
    "nonEmptyString": {
      "type": "string",
      "minLength": 1
     }
  }
}
    \end{minted}
    \caption{Simple JSON schema after allOf resolving}
    \label{listing:resolved-allOf}
\end{listing}

\paragraph{anyOf and oneOf} % todo

\paragraph{Title inducing}

The \texttt{title} keyword is used to give a schema a short description.
This is not necessarily the same as the property name of properties of an object.
As we use the title in various cases to display for the user, we inject the property name in cases where no explicit title is given.

\begin{listing}[!h]
    \begin{minted}[frame=single,
        framesep=3mm,
        linenos=true,
        xleftmargin=15pt,
        tabsize=4]{js}
{
  "type": "object",
  "properties": {
    "name": {
        "type": "string"
    }
  }
}
    \end{minted}
    \caption{Simple JSON schema with one property without a title}
    \label{listing:no-title}
\end{listing}

\begin{listing}[!h]
    \begin{minted}[frame=single,
        framesep=3mm,
        linenos=true,
        xleftmargin=15pt,
        tabsize=4]{js}
{
  "type": "object",
  "properties": {
    "name": {
        "title": "name",
        "type": "string"
    }
  }
}
    \end{minted}
    \caption{The property names was used for the title field}
    \label{listing:with-title}
\end{listing}

\paragraph{Processing enum and const}
The enum keyword is used to restrict the values of a field to a fixed set of valid values.
The const keyword, similarly, restricts the property value to a single allowed value.
Thus, setting the const value is equivalent to settings the enum value with an array that contains this single value.

We convert any usage of const to enums with a single element, which allows us to ignore the const keyword in other operations.

\subsubsection{Calculating an effective schema}

This third preprocessing step is calculated the most often, namely every time the data changes.
However, for most schemas this preprocessing step is trivial.
The JSON schema keywords \texttt{if}, \texttt{then}, and \texttt{else} provide a way to include conditions in the JSON schema.
If the schema in the \texttt{if} field is valid, then also the schema in the \texttt{then} field must be valid, otherwise the
schema in the \texttt{else} field must be valid.

This makes the schema data dependent.
To show the correct properties, we evaluate the data and dependent on validity or not, we either use the \texttt{then} or the \texttt{else} schema.

We similarly handle the \texttt{dependentRequired} and the \texttt{dependentSchemas} keywords.
For schemas without any of those keywords, this step is trivial as the schema is not modified in any way.

%todo examples

\subsubsection*{\bf A plain unnumbered list}
\begin{list}{}{}
    \item{bare\_jrnl.tex}
    \item{bare\_conf.tex}
    \item{bare\_jrnl\_compsoc.tex}
    \item{bare\_conf\_compsoc.tex}
    \item{bare\_jrnl\_comsoc.tex}
\end{list}

\subsection{Figures}
Fig. 1 is an example of a floating figure using the graphicx package.
Note that $\backslash${\tt{label}} must occur AFTER (or within) $\backslash${\tt{caption}}.
For figures, $\backslash${\tt{caption}} should occur after the $\backslash${\tt{includegraphics}}.

%\begin{figure}[!t]
%\centering
%\includegraphics[width=2.5in]{fig1}
%\caption{Simulation results for the network.}
%\label{fig_1}
%\end{figure}

Fig. 2(a) and 2(b) is an example of a double column floating figure using two subfigures.
(The subfig.sty package must be loaded for this to work.)
The subfigure $\backslash${\tt{label}} commands are set within each subfloat command,
and the $\backslash${\tt{label}} for the overall figure must come after $\backslash${\tt{caption}}.
$\backslash${\tt{hfil}} is used as a separator to get equal spacing.
The combined width of all the parts of the figure should do not exceed the text width or a line break will occur.
%
%\begin{figure*}[!t]
%\centering
%\subfloat[]{\includegraphics[width=2.5in]{fig1}%
%\label{fig_first_case}}
%\hfil
%\subfloat[]{\includegraphics[width=2.5in]{fig1}%
%\label{fig_second_case}}
%\caption{Dae. Ad quatur autat ut porepel itemoles dolor autem fuga. Bus quia con nessunti as remo di quatus non perum que nimus. (a) Case I. (b) Case II.}
%\label{fig_sim}
%\end{figure*}

Note that often IEEE papers with multi-part figures do not place the labels within the image itself (using the optional argument to $\backslash${\tt{subfloat}}[]), but instead will
reference/describe all of them (a), (b), etc., within the main caption.
Be aware that for subfig.sty to generate the (a), (b), etc., subfigure
labels, the optional argument to $\backslash${\tt{subfloat}} must be present. If a
subcaption is not desired, leave its contents blank,
e.g.,$\backslash${\tt{subfloat}}[].

 \section{User Study}\label{sec:user_study}
 Our user study is mainly based on the qualitative research.
We conduct several interviews with the professors and students as potential customers.
We collect feedbacks from the customers, improve user experiences and fix bugs according to the feedback.
We also use the survey to get interviewers' opinions during the user study.

\subsection{Methodology}
% todo: describe more details about how we conduct user study

\subsection{Results}
% todo: show the results of user study. We can use tables to show specific cases.

\subsection{Evaluation} %keyuri

\begin{table*}
    \vspace{-10pt}
    \centering
    \small % Reduce font size
    \setlength{\extrarowheight}{5pt} % Add extra vertical space
    \renewcommand{\arraystretch}{1.5} % Adjust the vertical spacing between rows
    \begin{tabular}{|c|p{0.4\linewidth}|p{0.4\linewidth}|}
        \hline
        \multicolumn{1}{|c|}{\multirow{2}{*}{}} & \multicolumn{1}{|c|}{Feedback} & \multicolumn{1}{|c|}{Resolution} \\
        \cline{2-3}
        \hline
        User Study 1 & 1. The property value should not be autocorrected if the user enters an incorrect value. Instead, an error message or another way should be used to inform the user that their input is incorrect. & 1. To enhance user feedback, consider implementing a feature that displays a red underline beneath the property value when an error is detected. \\
        \cline{2-3}
        & 2. The ability to delete a property should be made easily accessible and user-friendly within the GUI panel. & 2. The deletion functionality has been implemented by adding a cross symbol next to each property, allowing users to easily delete a property by clicking on the cross symbol. \\
        \cline{2-3}
        & 3. Including a search functionality is essential to help users locate properties, especially within nested levels where finding specific property can be challenging. & 3.A search button has been implemented in the top toolbar to enable users to locate properties, even within nested levels. When a property is found, it is highlighted in yellow to make it easily identifiable. \\
        \cline{2-3}
        & 4. The Schema editor GUI panel contains extensive metadata, but the code editor panel remains empty. & 4. bla bla bla \\
        \cline{2-3}
        & 5.To maintain consistency and readability, it should be remove the use of different font styles of property (bold, purple colour, * with purple colour and so on)  in the file editor within the GUI panel. & 5. bla bla bla \\
        \cline{2-3}
        & 6. The cursor styling in the file editor within the GUI panel should be improved to ensure it doesn't resemble a link cursor. & 6. Cursor styling is chnaged. \\
        \cline{2-3}
        & 7. The cross symbol has been removed from the dropdown button for enum properties in the GUI panel as it is deemed unnecessary. & 7. The cross symbol has been removed from the dropdown button for enum properties in the GUI panel. \\
        \cline{2-3}
        & 8. If the type of a property is as "any" in the Schema Editor, it should not default to being interpreted as the "string" type. & 8. bla bla \\
        \cline{2-3}
        & 9. Validation error should not showed as warning symbol, proper indication needed for validation error rather than warning symbol. & 9.The warning symbol has been removed, and a red cross indication has been implemented to clearly indicate validation errors in the user interface. \\
        \cline{2-3}
        & 10. Validation error should not showed as warning symbol, proper indication needed for validation error rather than warning symbol. & 10.The warning symbol has been removed, and a red cross indication has been implemented to clearly indicate validation errors in the user interface. \\
        \cline{2-3}
        & 11. After performing an undo or redo action, the cursor should jump to the appropriate location to reflect the changes made by the user.& 10. Will be consider for future work. \\
        \hline
    \end{tabular}
    \caption* {User Study Feedback and Resolution (Continued)}
\end{table*}


\clearpage % Force a page break

\begin{table*}

    \centering
    \small % Reduce font size
    \setlength{\extrarowheight}{5pt} % Add extra vertical space
    \renewcommand{\arraystretch}{1.5} % Adjust the vertical spacing between rows
    \begin{tabular}{|c|p{0.4\linewidth}|p{0.4\linewidth}|}
        \hline
        User Study 2 & 1. A graph-based approach would be more intuitive for handling complex settings. & 1. Will be considered for future work. For now, we have a tree-based approach rather than a graph-based one. \\
        \cline{2-3}
        & 2. Providing immediate feedback to users when they enter incorrect ranges is essential to prevent them from inputting invalid values into the property. & 2. Will be consider for future work. For now, a red underline serves as an indicator to direct the user's attention when something is wrong. Additionally, when hovering over it, a dialog box can be used to inform the user of specific conditions. \\
        & 3. Validation errors should also be reflected in the GUI panel, including for child properties. & 3. Currently, invalid properties are indicated in the code panel using a red cross mark, while in the GUI editor panel, invalid properties are highlighted with a red underline. \\
        \cline{2-3}
        & 4. When dealing with an array, the naming format for object name should be improved, replacing the default numerical labels (0, 1, 2, etc.) with a more descriptive way of showing the object. & 4. Improvement can be achieved by replacing simple numerical labels (0, 1, 2, etc.) for objects with a naming convention like "PropertyName[0]" and so forth. \\
        \cline{2-3}
        & 5. Additionally, when clicking a new property, it should not automatically create a new field, especially when there is already one field by default for adding the first property. & 5. Will be Consider for future work. \\
        \cline{2-3}
        & 6. In the schema editor, it would be more consistent and user-friendly if the functionality to give a name to a property were located on the right side of the GUI editor panel, eliminating the need to click on the property first in order to rename it. & 6. We have retained the property naming on the right end but improved it with an intuitive design. When clicking on a new property in the schema editor, it is now highlighted to prompt the user to change the name. \\
        \cline{2-3}
        & 7. The search function for locating specific properties lacks clarity at first glance. It should provide an immediate response and extend to nested levels, rather than merely highlighting the higher-level property in yellow. & 7. The search functionality has been enhanced. It now provides a list of properties containing the keyword, and upon clicking, it highlights the selected property while opening the nested levels in the GUI editor panel for improved usability. \\
        \hline
    \end{tabular}
    \caption*{User Study Feedback and Resolution (Continued)}
\end{table*}

\begin{table*}
    \centering
    \small % Reduce font size
    \setlength{\extrarowheight}{5pt} % Add extra vertical space
    \renewcommand{\arraystretch}{1.5} % Adjust the vertical spacing between rows
    \begin{tabular}{|c|p{0.4\linewidth}|p{0.4\linewidth}|}
        \hline
        User Study 3 & 1.Still I have to right & 1. bla bla \\
        \cline{2-3}

        \hline
    \end{tabular}
    \caption*{User Study Feedback and Resolution (Continued)}
\end{table*}

\begin{table*}

    \centering
    \small % Reduce font size
    \setlength{\extrarowheight}{5pt} % Add extra vertical space
    \renewcommand{\arraystretch}{1.5} % Adjust the vertical spacing between rows
    \begin{tabular}{|c|p{0.4\linewidth}|p{0.4\linewidth}|}
        \hline
        User Study 4 & 1.Modifying or renaming a 'new property' in Schema Editor does not appear to take effect when double-clicking on it. & 1. Renaming of the new property is made possible by double-clicking on 'newProperty' itself. \\
        \cline{2-3}
        & 2. Upon selecting a property type in the schema editor, the type of the child property should be automatically chnaged. & 2. The type of the child element is automatically adjusted according to parent type. \\
        \cline{2-3}
        & 3.A toggle button should be implemented to enable and disable the code panel and GUI panel. & 3. Will be consider for the future work. \\
        \cline{2-3}
        & 4.When working with a single property in the File Editor within the GUI Editor panel, the opacity of already open properties should be decreased, rather than making everything visible. & 3. Will be consider for the future work. \\
        \cline{2-3}
        & 5. Simplify the JSON schema, for those who are not very familiar with JSON schema. & 3. Will be consider for the future work. \\
        \hline

    \end{tabular}
    \caption{User Study Feedback and Resolution (Continued)}
\end{table*}

\begin{table*}

    \centering
    \small % Reduce font size
    \setlength{\extrarowheight}{5pt} % Add extra vertical space
    \renewcommand{\arraystretch}{1.5} % Adjust the vertical spacing between rows
    \begin{tabular}{|c|p{0.4\linewidth}|p{0.4\linewidth}|}
        \hline
        User Study 5 & 1. The search button is not immediately evident, making it challenging for users to locate the search function. & 1. We removed the search icon and made it expandable as an InputText field to help users find the desired property more easily. \\
        \cline{2-3}
        & 2. Should be added a toggle button to enable or disable either the GUI panel or the Code panel, providing users with the option to choose their preferred view. & 2. According to our vision, having only one view can be challenging when working with configuration files. However, we will consider this for future development. \\
        \hline

    \end{tabular}
    \caption{User Study Feedback and Resolution (Continued)}
\end{table*}

















% TODO: if some quantitative methods are found, also mention here

 \begin{table}[!t]
  \caption{An Example of a Table\label{tab:table1}}
  \centering
  \begin{tabular}{|c||c|}
   \hline
   One   & Two  \\
   \hline
   Three & Four \\
   \hline
  \end{tabular}
 \end{table}


 \section{Conclusion}\label{sec:conclusion}
 The conclusion goes here.


 \section*{Acknowledgments}
 This should be a simple paragraph before the References to thank those individuals and institutions who have supported your work on this article.



 \appendix[Proof of the Zonklar Equations]



%{\appendices
%\section*{Proof of the First Zonklar Equation}
%Appendix one text goes here.
% You can choose not to have a title for an appendix if you want by leaving the argument blank
%\section*{Proof of the Second Zonklar Equation}
%Appendix two text goes here.}



 \bibliographystyle{IEEEtran}
 \bibliography{literature}
 % argument is your BibTeX string definitions and bibliography database(s)
%\bibliography{IEEEabrv,../bib/paper}
%


%\begin{thebibliography}{1}
%\bibliographystyle{IEEEtran}


%\end{thebibliography}


 \newpage



 \vspace{11pt}

 \textbf{If you include a photo:}\vspace{-33pt}
%\begin{IEEEbiography}%[{\includegraphics[width=1in,height=1.25in,clip,keepaspectratio]{fig1}}]{Michael %Shell}
%Use $\backslash${\tt{begin\{IEEEbiography\}}} and then for the 1st argument use %$\backslash${\tt{includegraphics}} to declare and link the author photo.
%Use the author name as the 3rd argument followed by the biography text.
%\end{IEEEbiography}

 \vspace{11pt}

 \textbf{If you will not include a photo:}\vspace{-33pt}
 \begin{IEEEbiographynophoto}{John Doe}
  Use $\backslash${\texttt{begin\{IEEEbiographynophoto\}}} and the author name as the argument followed by the biography text.
 \end{IEEEbiographynophoto}




 \vfill

\end{document}


