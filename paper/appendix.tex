
\appendices\section{User Study}\label{sec:user-study}  % Minye
\subsection{Interview Tasks}\label{subsec:tasks}
This part is about the newest version of interview tasks we prepared to conduct our user study.\\
\textbf{Remark:} Task 3 was added after the first user study and also some details in other tasks were modified.
\subsubsection{Introduction}
For these tasks you are presented a schema that you have not seen or worked with before.
We have prepared a schema of a made-up simulation software in which self-driving cars are simulated.
There is one self-driving car that has to navigate from a start point to an end point but there are also other vehicles and pedestrians simulated.
For these tasks, the exact details are not important.

\subsubsection{Task 1: Setup}
\begin{enumerate}
    \item Go to https://paulbredl.github.io/config-assistant/
    \item Select the option ``Select a Schema'': ``Example schema'', then ``Autonomous Vehicle Schema".
    \item Open the example configuration file we have sent to you. \\
          The following tasks will assume this schema and this example file.
\end{enumerate}

\subsubsection{Task 2: Questions}
\begin{enumerate}
    \item What is the name of the simulation?
    \item What is the weather in our simulated environment?
    \item What is the total duration of the simulation?
    \item Humidity is a subproperty of the Environment.
    Would 150 be a valid value for Humidity?
\end{enumerate}

\subsubsection{Task 3: Modifying the configuration file}
\begin{enumerate}
    \item Change the name of the simulation to Sim\_Advanced05.
    \item The VehicleType of the self-driving car is currently ``Level 3''.
    Change it to the highest possible level.
    \item The configuration file has validation errors, i.e., it is not valid according to the schema.
    \\ Find out what the errors are.
    Edit the file so it becomes a valid configuration.

\end{enumerate}

\subsubsection{Task 4: Modifying the schema}
\begin{enumerate}
    \item For many applications it is good practice or even required that a schema has a unique identifier, which usually is a URL.
          \\Set the \$id field of this schema to : ``https://www.example.com/self-driving-vehicle''.
    \item The simulation software will get a premium version in the next update, for which the user needs a license.
          The license key should be stored in the configuration file.
          Add a new property ``LicenseKey'' to the ``properties'' object.
          It should be of the type ``string''.
          The length of the key is at most 20 characters long.
    Add a short exemplary description.
    \item Go back to the file editor and verify that the new property ``LicenseKey'' is displayed and that the correct information is displayed when hovering over the property.
\end{enumerate}





% Minye

\subsection{User Study results}
\begin{table*}[!htbp]
    \centering
    \caption{Results of user study} \label{table:results}
    \begin{tabular}{lp{4.5cm}p{6cm}}
        \toprule
        &
        \thead{Accuracy \& Notes} & \thead{Difficulties} \\
        \midrule

        User Study 1 &
            Accuracy: 100\%(11/11)\newline
            Effective use of tooltips\newline
            Used both GUI and code editor

            &

            Task 3.2: Setting Vehicle Type to the highest level:\newline
            Took some time to find the field\\ \midrule

        User Study 2 &
            Accuracy: 91\%(10/11)\newline
            Used search functionality\newline
            Used both GUI and code editor

            &

            Task 4.2: Adding new property:\newline
            Could not find where to add new property in Schema Editor\newline
            Did not know how to edit property name \\ \midrule

        User Study 3 &
            Accuracy: 82\%(9/11)\newline
            Effective use of tooltips\newline
            Not familiar with JSON schema\newline
            Used only code editor in beginning and then only GUI editor after task 2.3.

            &

            Task 2.3: Duration of Simulation:\newline
            Did not think about using the GUI panel to retrieve information\newline
            Task 2.4: Validity of humidity:\newline
            Mistakenly thought humidity value was valid.\newline
            Did not consider red underline as an error\newline
            Task 4.2: Add new property:\newline
            Set property name in wrong place
         \\ \midrule

        User Study 4 &
            Accuracy: 82\%(9/11)\newline
            Solved tasks in a short time\newline
            Used only the GUI editor

            &

            Task 2.4: Validity of humidity:\newline
            Mistakenly thought humidity value was valid\newline
            Did not find out that he could use tooltips\newline
            Task 3.2: Setting Vehicle Type to the highest level: \newline
            Did not scroll down in Dropdown menu\newline
            Task 3.3: Validation Errors:\newline
            Thought red underline was spell checking \\ \midrule

        User Study 5 &
            Accuracy: 100\%(11/11)\newline
            Solved tasks in a short time\newline
            Used both GUI and code editor

            &

            Task 3.2: Setting Vehicle Type to the highest level: \newline
            Took some time to find the vehicle type \newline
            Task 4.2: Adding new property:\newline
        Did not set the property type at the beginning \\ \bottomrule
    \end{tabular}
\end{table*}


\begin{table*}[!htbp]
    \centering
    \caption{User Study 1 - Feedback and Solution}
    \label{tab:user_study1}
    \begin{tabular}{p{0.45\linewidth}p{0.45\linewidth}}
        \toprule
         \thead{Feedback} & \thead{Solution} \\
        \midrule
        The property value should not be autocorrected if the user enters an incorrect value.
        Instead, an error message or another way should be used to inform the user that their input is incorrect.
        &
        Instead of autocorrecting values, we now provide more clear user feedback on incorrect values (red underline, error symbol). \\
        \midrule
        It would be good to have the ability to remove data entries with the GUI panel.
        &
        Implemented by adding a \textit{remove} button next to properties which have data and are not required. \\
        \midrule
        A search functionality to locate properties would be helpful, especially within nested levels.
        & Implemented in the toolbar.
        All findings are highlighted in the GUI panel. \\
        \midrule
        %& 4. The Schema editor GUI panel contains extensive metadata, but the code editor panel remains empty. & 4. bla bla bla \\
        %\midrule
        The GUI panel feels overwhelming to the user due to many variations in styling and color of the GUI elements.
        &
        We slightly reduced the number of different styling by no longer showing required properties in bold face and instead just show an asterisk next to it. \\
        \midrule
        The cursor should not have the clickable animation when hovering over non-clickable fields in the GUI editor.
        &
        Now we only show the clickable animation when hovering over clickable GUI components. \\
        \midrule
        In drop-down menus we do not need a button to clear the selection.
        &
        We disabled the option of clearing the selection. \\
        \midrule
        If the type of a property is ``any'', it should not be interpreted as the ``string'' type in the GUI panel.
        &
        We show a drop-down menu to the user, where they can select the type they want to use. \\
        \midrule
        Validation errors should not be highlighted via a warning symbol, but instead an error symbol.
        &
        We changed the warning symbol into an error symbol. \\
        \midrule
        After performing an undo or redo action, the cursor should jump to the corresponding location to reflect the changes made by the user.
        &
        Will be considered in future work. \\
        \bottomrule
    \end{tabular}
\end{table*}


\begin{table*}

    \centering
    \caption{User Study 2 - Feedback and Solution}
    \label{tab:user_study2}
    \begin{tabular}{p{0.45\linewidth}p{0.45\linewidth}}
        \toprule
        \thead{Feedback} & \thead{Solution} \\
        \midrule
        A graph-based view would be more intuitive for handling complex data structures.
        &
        Will be considered in future work. \\
        \midrule
        Providing immediate feedback to users when they enter incorrect ranges is essential to prevent them from inputting invalid values into the property.
        &
        We now highlight schema violations by a red error symbol in the GUI panel and underlining the property name in red.
        Additionally, the tooltip lists all schema violations of a property. \\
        \midrule
        Validation errors should also be reflected in the GUI panel, including for child properties.
        &
        See point above.
        Also, now the tooltip lists schema violations of child properties. \\
        \midrule
        When dealing with an array, the display name of array elements (index) should be improved.
        Currently, the tool only shows just the element index.
        &
        We replaced the numerical labels by a standard programming notation, which is \texttt{propertyName[0]}, \texttt{propertyName[1]}, \ldots
        \\
        \midrule
        The input field next to the \textit{Add Item} button is confusing.
        Both the input field and the button can be used to create a new item, which is redundant.
        &
        We removed the input field next to the button. \\
        \midrule
        It would be more consistent, if all user input in the GUI panel was within the right column of the table.
        That in some scenarios user input is needed within the left column (for names of new properties) feels inconsistent.
        &
        Because of the nature of JSON schema, we retained the property name within the left column.
        To make it clear to the user that the property name can be edited, we added an \textit{edit} icon next to it. \\
        \midrule
        The search function for locating specific properties lacks clarity at first glance.
        It should provide an immediate response and extend to nested levels, rather than merely highlighting the higher-level findings.
        &
        The search now provides a list of results, and upon clicking on a particular result, it jumps to that result in the code panel and GUI panel.
        In the GUI panel, if the element is nested, its parents will be automatically expanded. \\
        \bottomrule
    \end{tabular}

\end{table*}

\begin{table*}
    \centering
    \caption{User Study 3 - Feedback and Solution} \label{tab:user_study3}
    \begin{tabular}{p{0.45\linewidth}p{0.45\linewidth}}
        \toprule
        \thead{Feedback} & \thead{Solution} \\
        \midrule
        Working with the schema editor is difficult for me.
        It does not feel intuitive.
        &
        We made the schema editor more intuitive by creating our own simplified JSON schema meta schema.
        For example, advanced JSON schema features are separated from the simple ones.
        See section~\ref{subsec:schema-editor} \\
        \bottomrule

    \end{tabular}

\end{table*}

\begin{table*}
    \centering
    \caption{User Study 4 - Feedback and Solution} \label{tab:user_study4}
    \begin{tabular}{p{0.45\linewidth}p{0.45\linewidth}}
        \toprule
        \thead{Feedback} & \thead{Solution} \\
        \midrule
        Modifying or renaming a new property in the GUI panel does not appear to take effect when double-clicking on it.
        &
        Renaming properties in the GUI panel can now be done using the \textit{edit} button next to the property name. \\
        \midrule
        When creating a new property in the schema editor, its sub-schema has to be selected, such as \textit{string property} or \textit{boolean property}.
        Additionally, the type of the property has to be selected by the user too.
        Therefore, for example, when creating a new \textit{string property}, the user has to select that it is a string two times.
        It would be much more intuitive if the selection needs to be done only one time.
        & We completely overhauled our JSON schema meta schema.
        Now, when creating a new property, the user will have to select the type only once. \\
        \midrule
        A toggle button should be implemented to enable and disable the code panel and GUI panel.
        & Only having a GUI panel or only having a code panel restricts the user unnecessarily.
        The interplay of both panels is what makes this tool most effective.
         If the user does not want to use one of the panels, they can resize that panel to a very small size.  \\
        \midrule
        When working with a particular property in the GUI panel the opacity of the other properties should be decreased, visually highlighting the property that currently is in focus. & Will be considered in future work. \\
        \midrule
        Simplify the schema editor to make it easier to work with, for those who are not very familiar with JSON schema.
        &
        Has been done, see section~\ref{subsec:schema-editor}. \\
        \bottomrule

    \end{tabular}

\end{table*}

\begin{table*}

    \centering
    \caption{User Study 5 - Feedback and Solution} \label{tab:user_study5}
    \begin{tabular}{p{0.45\linewidth}p{0.45\linewidth}}
        \toprule
        \thead{Feedback} & \thead{Solution} \\
        \midrule
        The search button is not immediately evident, making it challenging for users to locate the search function. &
        Instead of showing the search bar only when clicking the search button, we now always show it. \\
        \bottomrule

    \end{tabular}

\end{table*}
