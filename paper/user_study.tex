We conduct several interviews with professors and students as potential customers.
We collect feedback from the customers to improve user experiences and fix bugs according to the feedback.
We also use a survey for demographic questions after the user study.

\subsection{Methodology}\label{subsec:methodology}

\paragraph{Potential Users}
We look for potential users to conduct the interviews.
There are two basic standards we follow to find a potential user:
\begin{itemize}
    \item Professors and students who are interested in our application.
    \item People who frequently use \cfgfiles and schema languages
\end{itemize}

\paragraph{Interview Questions}
We prepare our interview questions based on a mockup JSON schema and \cfgfiles.
It is about a simulation software that simulates a self-driving car.
To make it easier for interviewees to solve the questions, we divide our interview into four parts:
\begin{itemize}
    \item Setup Question: Where to open the application and set up schema and \cfgfiles.
    \item Basic Question: Goal is to know about the GUI of application
    \item Modifying the configuration file: Solving problems of \cfgfiles inside the \textbf{file editor}.
    \item Modifying the schema: Solving problems of JSON schema inside the \textbf{schema editor}.
\end{itemize}

\paragraph{Interview Process}
We proceed our interview with one interviewee each time.
The interview lasts around one hour.
At the beginning, the interviewee is asked about approval of recording, and they can stop participating at any time.
One of our team member hosts the interview and guides interviewee how to use our application based on the interview question.
Another one will record the whole interview and takes notes of what are correctly and incorrectly answered.
Opinions and critical points will also be noted.
After the application test, all team members can ask the open questions and answer questions from the interviewee.

\paragraph{Survey}
After the interview, we ask each interviewee to finish a survey(follow-up questions).
This survey will cover the following points:
\begin{itemize}
    \item Occupation of the interviewee.
    \item Experiences in software engineering,
    \item Frequency of working with JSON/YAML files.
    \item Domains in which people use with JSON/YAML files.
    \item Feedback about the application after interview.
\end{itemize}
Based on the survey, we could get many feedbacks which we can use in the future.
We analyse the results which we get from the survey and improve our application afterward.

\subsection{Results}
To evaluate the user study, we first evaluate which tasks the participants could solve correctly and which not.
We conducted five user studies in total and summarize the correctness and incorrectness of answers into the table, as shown in \ref{table:Results}.
The task is marked as wrong when the first answer is wrong , then this will be categorized into \textbf{Incorrect Answer}.
If the interviewee solved the task with some efforts or took more time, then it will be categorized into \textbf{Done with Efforts}
\begin{table*}
    \label{table:Results}
    \begin{tabular}{|c|ll|ll|}
        \hline
        \multicolumn{1}{|l|}{} &
        \multicolumn{2}{c|}{
            \begin{tabular}[c]{@{}c@{}}
                Correct\\
                Answer
            \end{tabular}} &
        \multicolumn{2}{c|}{

            \begin{tabular}[c]{@{}c@{}}
                Incorrect \\
                Answer
            \end{tabular}} \\ \hline

        \multicolumn{1}{|l|}{} &
        \multicolumn{1}{c|}{
            Accuracy \& Highlights} &
        \multicolumn{1}{c|}{

            \begin{tabular}[c]{@{}c@{}}
                Done\\
                with efforts
            \end{tabular}} &

        \multicolumn{2}{l|}{} \\ \hline
        User Study 1 &
        \multicolumn{1}{l|}{

            \begin{tabular}[c]{@{}l@{}}Accuracy: 100\%(11/11)\\ 1. Effectively used tooltip\\ 2. All questions right
            \end{tabular}} &

        \begin{tabular}[c]{@{}l@{}}1. Task 3.2: Setting Vehicle Type to highest level\\
        - Took some time to find the field; \\
        - Not used the search function
        \end{tabular} &
        \multicolumn{2}{c|}{none} \\ \hline
        User Study 2 &
        \multicolumn{1}{l|}{

            \begin{tabular}[c]{@{}l@{}}Accuracy: 91\%(10/11)\\
            1. Tried to find search function\\
            to solve task
            \end{tabular}} &

        \begin{tabular}[c]{@{}l@{}}1. Task 3.3: Validation Errors\\
        - Expected obvious error hint\\
        - Not used the search function
        \end{tabular} &

        \multicolumn{2}{l|}{
            \begin{tabular}[c]{@{}l@{}}1.Task 4.2: Adding new property\\
            - Firstly added property on root level\\
            - Not knew how to change property name
            \end{tabular}} \\ \hline

        User Study 3 &
        \multicolumn{1}{l|}{
            \begin{tabular}[c]{@{}l@{}}Accuracy: 82\%(9/11)\\
            1. used tooltip to solve the tasks\\
            \end{tabular}} &

        \begin{tabular}[c]{@{}l@{}}1. Task 2.3: Duration of Simulation\\
        - Not knew one can use GUI panel \\
        at the beginning\\
        2. Task 3.3: Validation Errors\\
        - Not found where the error locates(Spell Error)
        \end{tabular} &

        \multicolumn{2}{l|}{
            \begin{tabular}[c]{@{}l@{}}1. Task 2.4: Valid humidity value\\
            - Ignored the error hint\\
            3. Task 4.2 Add new property\\
            - Set property name in wrong place
            \end{tabular}} \\ \hline

        User Study 4 &
        \multicolumn{1}{l|}{
            \begin{tabular}[c]{@{}l@{}}Accuracy: 82\%(9/11)\\
            1. Knew about JSON schema already\\
            2. Added new property quickly
            \end{tabular}} &

        \begin{tabular}[c]{@{}l@{}}1. Task 3.3: Validation Errors\\
        - Tried to find validation errors in schema editor\\
        - Thought red underline was spell checking
        \end{tabular} &

        \multicolumn{2}{l|}{
            \begin{tabular}[c]{@{}l@{}}1. Task 2.4: Valid value of humidity\\
            - Not found that value was invalid\\
            - Not found that he could use tooltip\\
            2. Task 3.2: Setting Vehicle Type to highest level \\
            - Not scrolling down in enum property
            \end{tabular}} \\ \hline

        User Study 5 &
        \multicolumn{1}{l|}{
            \begin{tabular}[c]{@{}l@{}}Accuracy: 100\%(11/11)\\
            1. Understood the tool very well\\
            2. Quickly found all errors\\
            and solved them\\
            3. Knew to use both sides of tool
            \end{tabular}} &

        \begin{tabular}[c]{@{}l@{}}1. Task 3.2: \\
        - Took some time to find the vehicle type\\
        2. Task 4.2: Adding new property\\
        - Did not set the property type at the beginning
        \end{tabular} &
        \multicolumn{2}{c|}{none} \\ \hline
    \end{tabular}
\end{table*}

\subsection{Evaluation}
% todo: show correctness of the tasks.


% TODO: if some quantitative methods are found, also mention here