We conduct several interviews with professors and students as potential customers.
We collect feedback from the customers to improve user experiences and fix bugs according to the feedback.
We also use a survey for demographic questions after the user study.

\subsection{Methodology}\label{subsec:methodology}

\paragraph{Potential Users}
We look for potential users to conduct the interviews.
There are two basic standards we follow to find a potential user:
\begin{itemize}
    \item Professors and students who are interested in our application.
    \item People who frequently use \cfgfiles and schema languages
\end{itemize}

\paragraph{Interview Questions}
We prepare our interview questions based on a mockup JSON schema and \cfgfiles.
It is about a simulation software that simulates a self-driving car.
To make it easier for interviewees to solve the questions, we divide our interview into four parts:
\begin{itemize}
    \item Setup Question: Where to open the application and set up schema and \cfgfiles.
    \item Basic Question: Goal is to know about the GUI of application
    \item Modifying the configuration file: Solving problems of \cfgfiles inside the \textbf{file editor}.
    \item Modifying the schema: Solving problems of JSON schema inside the \textbf{schema editor}.
\end{itemize}

\paragraph{Interview Process}
We proceed our interview with one interviewee each time.
The interview lasts around one hour.
At the beginning, the interviewee is asked about approval of recording, and they can stop participating at any time.
One of our team member hosts the interview and guides interviewee how to use our application based on the interview question.
Another one will record the whole interview and takes notes of what are correctly and incorrectly answered.
Opinions and critical points will also be noted.
After the application test, all team members can ask the open questions and answer questions from the interviewee.

\paragraph{Survey}
After the interview, we ask each interviewee to finish a survey(follow-up questions).
This survey will cover the following points:
\begin{itemize}
    \item Occupation of the interviewee.
    \item Experiences in software engineering,
    \item Frequency of working with JSON/YAML files.
    \item Domains in which people use with JSON/YAML files.
    \item Feedback about the application after interview.
\end{itemize}
Based on the survey, we could get many feedbacks which we can use in the future.
We analyse the results which we get from the survey and improve our application afterward.

\subsection{Results}
To evaluate the user study, we conducted five user studies in total and summarize the biggest challenges they encountered, as shown in \ref{table:Results}.
All participants use the same task sheet , and they are given around 30 minutes to solve the tasks.
We firstly evaluate the background of participants and list all biggest challenges they encountered during the interview.
Accuracy is calculated as the tasks, which participants solved without any hints.
All participants were able to complete all tasks after we gave hints on those tasks where they were struggling.
\begin{table*}[]
    \label{table:Results}
    \begin{tabular}{|c|l|l|l|}
        \hline
        \multicolumn{1}{|l|}{} &
        \begin{tabular}[c]{@{}l@{}}Participant\\ Background\end{tabular} &
        \multicolumn{1}{c|}{
            \begin{tabular}[c]{@{}c@{}}Accuracy\\ \&\\ Notes
            \end{tabular}} &
        \multicolumn{1}{c|}{
            \begin{tabular}[c]{@{}c@{}}Biggest\\ Difficulties
            \end{tabular}} \\ \hline
        User Study 1 &
        \begin{tabular}[c]{@{}l@{}}Already familiar with\\ JSON Schema
        \end{tabular} &
        \begin{tabular}[c]{@{}l@{}}Accuracy: 100\%(11/11)\\
        1. Effectively solved tasks\\ with help of tooltip\\
        2. All questions right
        \end{tabular} &

        \begin{tabular}[c]{@{}l@{}}
            Task 3.2: Setting Vehicle Type to highest level\\
            - Took some time to find the field; \\
            - Not used the search function
        \end{tabular} \\ \hline
        User Study 2 &
        \begin{tabular}[c]{@{}l@{}}Already familiar with\\ JSON Schema\end{tabular} &
        \begin{tabular}[c]{@{}l@{}}Accuracy: 91\%(10/11)\\
        1. found search function\\     to solve task\\
        2.Use both GUI and code editor\\ to solve tasks
        \end{tabular} &
        \begin{tabular}[c]{@{}l@{}}
            Task 4.2: Adding new property\\
            - Could not find where to add new property\\ in Schema Editor\\
            - Did not know how to edit property name
        \end{tabular} \\ \hline
        User Study 3 &
        \multicolumn{1}{c|}{New to JSON Schema} &
        \begin{tabular}[c]{@{}l@{}}Accuracy: 82\%(9/11)\\
        1. Solved questions effectively \\ using tooltip\\
        2. Not familiar with JSON,\\ but could still solve tasks with \\ some help
        \end{tabular} &
        \begin{tabular}[c]{@{}l@{}}
            Task 2.3: Duration of Simulation\\
            - Did not know one can use GUI panel \\    at the beginning\\
            Task 2.4: Valid humidity error\\ - Could not find invalidity of humidity\\
            - Ignored the error hint at beginning\\
            Task 4.2 Add new property\\  - Set property name in wrong place
        \end{tabular} \\ \hline
        User Study 4 &
        \begin{tabular}[c]{@{}l@{}}Already familiar with \\ JSON Schema\end{tabular} &
        \begin{tabular}[c]{@{}l@{}}Accuracy: 82\%(9/11)\\
        1. Added new property quickly\\
        2. Resolved all tasks rightly
        \end{tabular} &
        \begin{tabular}[c]{@{}l@{}}
            Task 2.4: Valid value of humidity\\
            - Did not find value was invalid\\
            - Did not find that he could use tooltip\\
            Task 3.2: Setting Vehicle Type to highest level \\
            - Not scrolling down in enum property\\
            Task 3.3: Validation Errors\\
            - Tried to find validation errors in schema editor\\
            - Thought red underline was spell checking
        \end{tabular} \\ \hline
        User Study 5 &
        \begin{tabular}[c]{@{}l@{}}Seldom worked with\\ JSON schema\end{tabular} &
        \begin{tabular}[c]{@{}l@{}}Accuracy: 100\%(11/11)\\
        1. Understood the tool very well\\
        2. Quickly found all errors\\     and solved them\\
        3. Knew to use both GUI \\and code editor of tool
        \end{tabular} &
        \begin{tabular}[c]{@{}l@{}}
            Task 3.2: \\  - Took some time to find the vehicle type\\
            Task 4.2: Adding new property\\  - Did not set the property type at the beginning
        \end{tabular} \\ \hline
    \end{tabular}
    \caption{Results}
    \label{tab:my-table}
\end{table*}

\subsection{Evaluation}
% todo: show correctness of the tasks.


% TODO: if some quantitative methods are found, also mention here