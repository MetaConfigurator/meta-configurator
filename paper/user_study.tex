We conduct a qualitative user study with five participants.

% TODO: maybe somewhere mention demographics of user study? Like that it was 3 prof and 2 researcher?
During each interview, we introduce \toolname{}, give the participant tasks to execute using the tool and finish the session with open ended questions.
We observe how the participants work with the tool and which difficulties they have when executing the tasks.
Additionally, we ask them for feedback and improvement suggestions.
We also conduct a survey to gather demographic information about the participants of the user study.

Note that besides the user study we have applied \toolname{} on several schemas (such as EnzymeML\cite{TODO} and the Strenda schema\cite{TODO}) and \cfgfiles{} from the real world, to verify that it works and does bring benefits to the user.

\subsection{Research Questions}\label{subsec:research_questions}
We intend to address the following research questions with the user study:
\begin{enumerate}
	\item \textbf{RQ1:} Which aspects of the tool can be improved?
	\item \textbf{RQ2:} Are users able to perform the followings types of tasks using the tool:
	 \begin{itemize}
			\item \textbf{RQ2.1} Retrieve information from \cfgfiles{} in the context of a given schema
			\item \textbf{RQ2.2} Modify \cfgfiles{} within the constraints of a given schema
			\item \textbf{RQ2.3} Modify a schema file
		\end{itemize}
	\item \textbf{RQ3:} Would people use the tool in practice?
\end{enumerate}

\subsection{Methodology}\label{subsec:methodology} % Minye

\paragraph{Potential Users}
We look for potential users to conduct the interviews.
There are two basic standards we follow to find a potential user:
\begin{itemize}
    \item Professors and students who are interested in our application.
    \item People who frequently use \cfgfiles and schema languages
\end{itemize}

\paragraph{Interview Questions}
For the interview, we created a JSON schema and \cfgfile{}, about a made-up self-driving car simulation,
The interview questions deal with working with those files.
We divide our interview tasks into four parts:
\begin{itemize}
    \item Setup: Open the application and open schema and \cfgfile.
    \item Information Retrieval Questions: with increasing difficulty, the participant has to retrieve information from the \cfgfile{}.
    Much easier to solve by using the GUI panel.
    \item Configuration Modifications: Different tasks that involve changing the \cfgfile{}.
    Intend is that the participant becomes more familiar with the GUI panel.
    \item Schema modifications: Making adjustments to the schema.
    Most difficult, but still feasible using the GUI panel.
\end{itemize}

The interview tasks can be found in the appendix in section \ref{subsec:tasks}.

%TODO: include interview material in appendix

\paragraph{Interview Process}
We proceed our interview with one interviewee each time.
The interview lasts around one hour.
At the beginning, the interviewee is asked about approval of recording, and they can stop participating at any time.
We introduce \toolname{} to the interviewee.
Then we send the participant the tasks, an example \cfgfile{} and let them work on the tasks while sharing their screen.
During this task solving session, we provide the interviewee with some basic help if they ask specific questions about the tool.
The interview is recorded, and we make notes of the answers, feedback and behavior of the interviewee.
Finally, in an open dialog, we ask the interviewee about more feedback and improvement suggestions as well as their opinion on the tool.

\paragraph{Survey}
After the interview, we ask each interviewee to fill out a survey (follow-up questions).
This survey covers the following points:
\begin{itemize}
    \item Occupation of the interviewee.
    \item Experiences in software engineering,
    \item Frequency of working with JSON/YAML files.
    \item Domains in which people use with JSON/YAML files.
    \item Feedback about the application after interview.
\end{itemize}
%Based on the survey, we could get many feedbacks which we can use in the future.
%We analyse the results which we get from the survey and improve our application afterward.

\subsection{Results}\label{subsec:results}


\subsubsection{\textbf{RQ1:} Which aspects of the tool can be improved?} %Keyuri
Tables~\ref{tab:user_study1}-\ref{tab:user_study5} (in the appendix) show the feedback of the interviewees, as well as which measures we took based on it.

% TODO: Summarize results and answer the RQ


\subsubsection{\textbf{RQ2:} Are users able to perform the followings types of tasks using the tool?}
Table~\ref{table:results} (in the appendix) shows the accuracy and difficulties that the participants had when solving the tasks.
Accuracy is determined as the ratio of tasks that were solved correctly without the need of any hints to the total number of tasks.
With the help of hints, all tasks could be completed by every participant.
\begin{itemize}
    \item \textbf{RQ2.1}: Table~\ref{table:results} shows that all
    participants were able to retrieve information related to the schema and data.
    All participants were using mostly the GUI panel to retrieve information.
    Difficulties occurred when participants did not know that they could use the tooltips to get more information about the properties.
    \item \textbf{RQ2.2}: All participants were able to modify the given test file according to the given tasks.
    For some (User study 1 and 4) it was challenging to find a specific property, and they were not aware of the search functionality.
    \item \textbf{RQ2.3}: Most of the users are able to modify the schema file according to the tasks.
    The biggest difficulty was that participants did not know how to add a new property in the schema editor using the GUI panel.
\end{itemize}

Given that average accuracy is approximately 0.9, we conclude that the participants were able to perform all three types of tasks using \toolname{}.

\subsubsection{\textbf{RQ3:} Would people use the tool in practice?} % Keyuri



% TODO: create table regarding "Would you use the tool in practice?" with one row per participant
% Keyuri




% maybe add more interpretations of result

\subsection{Threads to Validity}\label{subsec:threads_to_validity} % Paul

Our user study is qualitative and only has a small sample size of 5 participants.
Hence, the results are not representative and cannot be generalized.
In addition, the tasks of the user study do only cover a small subset of operations
that a user may want to perform with \toolname{} to edit \cfgfiles{} and schemas.
Due to the time constraints of the user study, we only cover rather simple tasks.
Thus, for very complex tasks, the user study is not conclusive.
%The main purpose of the user study was to get feedback on the tool and to find out whether it is useful in practice,
% which we achieved.
For the purpose of getting feedback on the tool and finding out whether it is useful in practice, these limitations are acceptable.
For RQ2, we cannot generalize our results but only provide a first impression of how users work with \toolname{}.



