We conduct a qualitative user study with five participants.
During each interview, we introduce \toolname{}, give the participant tasks to execute using the tool and finish the session with open-ended questions.
We observe how the participants work with the tool and which difficulties they have when executing the tasks.
Additionally, we ask them for feedback and improvement suggestions.
We also conduct a survey to gather demographic information about the participants of the user study.

\subsection{Research Questions}\label{subsec:research_questions}
We address the following research questions with the user study:
\begin{enumerate}
	\item \textbf{RQ1:} Which aspects of the tool can be improved?
	\item \textbf{RQ2:} Are users able to perform the followings types of tasks using the tool:
	 \begin{itemize}
			\item \textbf{RQ2.1} Retrieve information from \cfgfiles{} in the context of a given schema
			\item \textbf{RQ2.2} Modify \cfgfiles{} within the constraints of a given schema
			\item \textbf{RQ2.3} Modify a schema file
		\end{itemize}
	\item \textbf{RQ3:} Would people use the tool in practice?
\end{enumerate}

\subsection{Methodology}\label{subsec:methodology} % Minye

Our user study is qualitative and consists of five interviews.
We do not perform any major quantitative analysis of the results as the sample size is too small to draw any statistically significant conclusions.

\subsubsection{Potential Users}
We look for potential users to conduct the interviews.
We consider the following groups of people as potential users:
\begin{itemize}
    \item Professors and students who are interested in our application.
    \item People who frequently use \cfgfiles{} and schema languages
\end{itemize}
In table~\ref{tab:user_study_feedback}, the professions of the participants are listed.

\subsubsection{Interview Questions}
We created a JSON schema and \cfgfile{}, about a made-up self-driving car simulation.
The interview questions deal with working with those files.
By using a made-up example, we avoid that the participants are biased by their domain knowledge.
We note, however, that besides the user study we have applied \toolname{} on several real-world schemas (such as EnzymeML\cite{pyenzyme} and the Strenda schema\cite{strenda}) and \cfgfiles{} from the real world, to verify that it works and does bring benefits to the user.

We divide our interview tasks into four parts:
\begin{itemize}
    \item Setup: We guide the participant through the setup of the tool, which involves accessing the tool via a web browser and loading the example files.
    \item Information Retrieval Questions: with increasing difficulty, the participant has to retrieve information from the \cfgfile{}.
    \item Configuration Modifications: The participant has to modify the given \cfgfile{} in various ways.
    \item Schema modifications: The participant has to modify the schema file.
\end{itemize}

The interview tasks and additional files can be found in out GitHub repository\footnote{\url{https://github.com/PaulBredl/config-assistant/tree/main/paper/userstudy}}.

\subsubsection{Interview Process}
We proceed our interview with one interviewee each time.
The interview lasts around one hour.
At the beginning, the interviewee is asked about approval of recording, and they can stop participating at any time.
We introduce \toolname{} to the interviewee.
Then we send the participant the tasks, an example \cfgfile{} and let them work on the tasks while sharing their screen.
During this task solving session, we provide the interviewee with some basic help if they ask specific questions about the tool.
The interview is recorded, and we make notes of the answers, feedback and behavior of the interviewee.
Finally, in an open dialog, we ask the interviewee about more feedback and improvement suggestions as well as their opinion on the tool.
After the interview, we ask each interviewee to fill out a survey with some questions about their background and their opinion on the tool.
%\paragraph{Survey}
%This survey covers the following points:
%\begin{itemize}
%    \item Occupation of the interviewee.
%    \item Experiences in software engineering,
%    \item Frequency of working with JSON/YAML files.
%    \item Domains in which people use with JSON/YAML files.
%    \item Feedback about the application after interview.
%\end{itemize}
%Based on the survey, we could get many feedbacks which we can use in the future.
%We analyse the results which we get from the survey and improve our application afterward.

\subsection{Results}\label{subsec:results}

\begin{table*}[!hbt]
    \centering
    \caption{User Study - Task solving accuracy and difficulties} \label{table:results}
    \begin{tabular}{lp{4.5cm}p{6cm}}
        \toprule
        &
        \thead{Accuracy \& Notes} & \thead{Difficulties} \\
        \midrule

        User Study 1 &
        Accuracy: 100\%(11/11)\newline
        Effective use of tooltips\newline
        Used both GUI and code editor

        &

        Task 3.2: Setting Vehicle Type to the highest level:\newline
        Took some time to find the field\\ \midrule

        User Study 2 &
        Accuracy: 91\%(10/11)\newline
        Used search functionality\newline
        Used both GUI and code editor

        &

        Task 4.2: Adding new property:\newline
        Could not find where to add new property in Schema Editor\newline
        Did not know how to edit property name \\ \midrule

        User Study 3 &
        Accuracy: 82\%(9/11)\newline
        Effective use of tooltips\newline
        Not familiar with JSON schema\newline
        Used only code editor in beginning and then only GUI editor after task 2.3.

        &

        Task 2.3: Duration of Simulation:\newline
        Did not think about using the GUI panel to retrieve information\newline
        Task 2.4: Validity of humidity:\newline
        Mistakenly thought humidity value was valid.\newline
        Did not consider red underline as an error\newline
        Task 4.2: Add new property:\newline
        Tried to set the property name at the wrong place
        \\ \midrule

        User Study 4 &
        Accuracy: 82\%(9/11)\newline
        Solved tasks in a short time\newline
        Used only the GUI editor

        &

        Task 2.4: Validity of humidity:\newline
        Mistakenly thought humidity value was valid\newline
        Did not find out that he could use tooltips\newline
        Task 3.2: Setting Vehicle Type to the highest level: \newline
        Did not scroll down in Dropdown menu\newline
        Task 3.3: Validation Errors:\newline
        Thought red underline was spell checking \\ \midrule

        User Study 5 &
        Accuracy: 100\%(11/11)\newline
        Solved tasks in a short time\newline
        Used both GUI and code editor

        &

        Task 3.2: Setting Vehicle Type to the highest level: \newline
        Took some time to find the vehicle type \newline
        Task 4.2: Adding new property:\newline
        Did not set the property type \\ \bottomrule
    \end{tabular}
\end{table*}
This section presents the results of the user study for each research question.

\subsubsection{\textbf{RQ1:} Which aspects of the tool can be improved?}\label{subsubsec:rq1} %Keyuri
Tables~\ref{table:user_study1}-\ref{table:user_study5} (in the appendix) show the feedback of the interviewees, as well as which measures we took based on it.


\subsubsection{\textbf{RQ2:} Are users able to perform the followings types of tasks using the tool?}
Table~\ref{table:results} shows the accuracy and difficulties that the participants had when solving the tasks.
Accuracy is determined as the ratio of tasks that were solved correctly without the need of any hints to the total number of tasks.
We note that all participants were able to solve all tasks after receiving hints from the interviewer.
%With the help of hints, all tasks could be completed by every participant.


\begin{itemize}
    \item \textbf{RQ2.1}: Table~\ref{table:results} shows that all
    participants were able to retrieve information related to the schema and data.
    All participants were using mostly the GUI panel to retrieve information.
    Difficulties occurred when participants did not know that they could use the tooltips to get more information about the properties.
    \item \textbf{RQ2.2}: All participants were able to modify the given test file according to the given tasks.
    For some (User study 1 and 4) it was challenging to find a particular property, and they were not aware of the search functionality.
    \item \textbf{RQ2.3}: Most of the users were able to modify the schema file according to the tasks.
    The biggest difficulty was that participants did not know how to add a new property in the schema editor using the GUI panel.
\end{itemize}

Given that average accuracy is approximately 0.9, we conclude that the participants were able to perform all three types of tasks using \toolname{}.

\subsubsection{\textbf{RQ3:} Would people use the tool in practice?} % Keyuri
Table~\ref{tab:user_study_feedback} shows the positive feedback that the participants provided and where they can imagine the tool to be used in practice.
Three of the participants highlighted the intuitiveness of \toolname.
Four of the participants explicitly describe the tool with the word ``useful'' or ``helpful''.
Additionally, all five participants responded that they themselves would use the tool.
Consequently, this result suggests that there is a demand for \toolname{} in practice.

Note that we only include positive feedback in this table since criticism is already addressed in section~\ref{subsubsec:rq1}.


\begin{table*}[htb] %keyuri, felix
    \centering
    \caption{User Study - Would people use the tool in practice?}\label{tab:user_study_feedback}
    \begin{tabular}{p{0.06\linewidth}p{0.1\linewidth}p{0.1\linewidth}p{0.3\linewidth}p{0.3\linewidth}}
        \toprule
        \textbf{User Study} & \textbf{Profession} & \textbf{Would you use this tool?} & \textbf{Potential Use case} & \textbf{Positive Feedback} \\
        \midrule
        1 & Professor & Yes & Configuring CI and simulations &
        ``[I think always having those two views (code panel and GUI panel) is very helpful].''

        ``[It was more comfortable to modify things using the GUI panel than the code panel].''

        ``You also showed nice examples.
        [Without the GUI] you would need to jump back to the schema to find out what is the max value.''
        \\
        \midrule
        2 & Professor & Yes & ``I would especially recommend [the tool] to some of my project partners.''

        ``I would certainly not only recommend but directly invite another partner to join in.'' &
        ``I think it is very intuitive already.''

        ``The search [functionality] is good.''

        ``I really like the tool and I think it's really very intuitive to work [with the GUI panel] and it's good to have the overview on the [code panel].
        The combination of both views is really helpful.
        Really good.
        I like it.''\\
        \midrule
        3 & Software Engineer & Yes & ``Currently I'm working with many experimental partners from different sort of chemistry and biological laboratories and the problem is that they just dump all data in some Excel sheet, and it's not machine-readable.
        [\ldots] Much data is lost since the only person who can work with the [data] is the person who created the Excel spreadsheet.
        Being able [\ldots] as a scientist [\ldots] to define what are your important properties, your schema, and then being able to [fill in the data] while running the experiment simultaneously, that would be a really, really huge advantage and the most prominent user that [comes to my mind].''
         &
        ``I understood how [the tool] works and I think it is really, really nice and will help many, many people in getting their data organized.''

        ``I think having this really clickable and intuitive way to show the structure of your schema is really, really neat, and it really looks pretty.''

        ``I really like where this project is heading.
            I always have this research field in mind and I think it will really have an impact on that, when we can promote it to our partners [\ldots] it will really make a difference.'' \\
        \midrule
        4 & Professor & Yes & Chemistry and biology field.
            Considers using the tool for their work.&
        We did not ask this participant for general feedback and only talked on a technical level. \\
        \midrule
        5 & Student & Yes & ``If I have to change more things [in a JSON document] it might be very nice to do it in an interface like this, especially given that it also provides the errors [\ldots], which otherwise you would probably only get in the command line.''

        ``I guess I'm a big fan, like even though I know how to use Git on a command line level, I often use the GUI [for several tasks] because it's just faster for many things.
        And I would say [your tool] is similar to that.''


        &
        ``[I was] positively surprised, like it was super easy to [\ldots] navigate and [for example] parsing JSON is always a bit of a pain, if I would have to go through this manually.
        [\ldots] It's like comparing Word and LaTeX. This felt basically like Word.
        I can just change things directly in an intuitive user interface.
        [\ldots] It is pretty nice, I would say.''

        ``Are you going to open source this?
        Because it might actually be useful.''\\
        \bottomrule
    \end{tabular}
\end{table*}




% maybe add more interpretations of result

\subsection{Threats to Validity}\label{subsec:threads_to_validity} % Paul

Our user study is qualitative and only has a small sample size of 5 participants.
Hence, the results are not representative and cannot be generalized.
In addition, the tasks of the user study do only cover a small subset of operations
that a user may want to perform with \toolname{} to edit \cfgfiles{} and schemas.
Due to the time constraints of the user study, we only cover rather simple tasks.
Thus, for very complex tasks, the user study is not conclusive.
%The main purpose of the user study was to get feedback on the tool and to find out whether it is useful in practice,
% which we achieved.
For the purpose of getting feedback on the tool and finding out whether it is useful in practice, these limitations are acceptable.
For RQ2, we cannot generalize our results but only provide a first impression of how users work with \toolname{}.



