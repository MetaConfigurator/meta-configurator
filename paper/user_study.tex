We conduct several interviews with professors and students as potential customers.
During each interview, we introduce \toolname{}, give the participant tasks to execute using the tool and finish the session with open ended questions.
We observe how the participants work with the tool and which difficulties they have when executing the tasks.
Additionally, we ask them for feedback and improvement suggestions.
We also conduct a survey to gather demographic information about the participants of the user study.

Note that besides the user study we have applied \toolname{} on several schemas (such as EnzymeML\cite{TODO} and the Strenda schema\cite{TODO}, provided by professors that we worked with) and \cfgfiles{} from the real world, to verify that it works and does bring benefits to the user.

\subsection{Research Questions}\label{subsec:research_questions}
We intend to address the following research questions with the user study:
\begin{enumerate}
	\item \textbf{RQ1:} Which aspects of the tool can be improved?
	\item \textbf{RQ2:} Are users able to perform the followings types of tasks using the tool:
	 \begin{itemize}
			\item \textbf{RQ2.1} Retrieve information from \cfgfiles{} in the context of a given schema
			\item \textbf{RQ2.2} Modify \cfgfiles{} within the constraints of a given schema
			\item \textbf{RQ2.3} Modify a schema file
		\end{itemize}
	\item \textbf{RQ3:} Would people use the tool in practice?
\end{enumerate}

\subsection{Methodology}\label{subsec:methodology} % Minye

\paragraph{Potential Users}
We look for potential users to conduct the interviews.
There are two basic standards we follow to find a potential user:
\begin{itemize}
    \item Professors and students who are interested in our application.
    \item People who frequently use \cfgfiles and schema languages
\end{itemize}

\paragraph{Interview Questions}
For the interview, we created a JSON schema and \cfgfile{}, about a made-up self-driving car simulation, which could be found in the appendix. \ref{subsec:tasks}
The interview questions deal with working with those files.
We divide our interview tasks into four parts:
\begin{itemize}
    \item Setup: Open the application and open schema and \cfgfile.
    \item Information Retrieval Questions: with increasing difficulty, the participant has to retrieve information from the \cfgfile{}.
    Much easier to solve by using the GUI panel.
    \item Configuration Modifications: Different tasks that involve changing the \cfgfile{}.
    Intend is that the participant becomes more familiar with the GUI panel.
    \item Schema modifications: Making adjustments to the schema.
    Most difficult, but still feasible using the GUI panel.
\end{itemize}

%TODO: include interview material in appendix

\paragraph{Interview Process}
We proceed our interview with one interviewee each time.
The interview lasts around one hour.
At the beginning, the interviewee is asked about approval of recording, and they can stop participating at any time.
We introduce \toolname{} to the interviewee.
Then we send the participant the tasks, an example \cfgfile{} and let them work on the tasks while sharing their screen.
During this task solving session, we provide the interviewee with some basic help if they ask specific questions about the tool.
The interview is recorded, and we make notes of the answers, feedback and behavior of the interviewee.
Finally, in an open dialog, we ask the interviewee about more feedback and improvement suggestions as well as their opinion on the tool.

\paragraph{Survey}
After the interview, we ask each interviewee to fill out a survey (follow-up questions).
This survey covers the following points:
\begin{itemize}
    \item Occupation of the interviewee.
    \item Experiences in software engineering,
    \item Frequency of working with JSON/YAML files.
    \item Domains in which people use with JSON/YAML files.
    \item Feedback about the application after interview.
\end{itemize}
%Based on the survey, we could get many feedbacks which we can use in the future.
%We analyse the results which we get from the survey and improve our application afterward.

\subsection{Results}\label{subsec:results}
We performed the user study on 5 participants.


\subsubsection{\textbf{RQ1:} Which aspects of the tool can be improved?} %Keyuri
Tables~\ref{tab:user_study1}-\ref{tab:user_study5} show the feedback of the interviewees, as well as which measures we took based on it.

% TODO: Summarize results and answer the RQ

\begin{table*}[!htbp]
    \centering
    \caption{User Study 1 - Feedback and Resolution}
    \label{tab:user_study1}
    \begin{tabular}{p{0.45\linewidth}p{0.45\linewidth}}
        \toprule
         \thead{Feedback} & \thead{Resolution} \\
        \midrule
        The property value should not be autocorrected if the user enters an incorrect value.
        Instead, an error message or another way should be used to inform the user that their input is incorrect.
        &
        Instead of autocorrecting values, we now provide more clear user feedback on incorrect values (red underline, error symbol). \\
        \midrule
        It would be good to have the ability to remove data entries with the GUI panel.
        &
        Implemented by adding a \textit{remove} button next to properties which have data and are not required. \\
        \midrule
        A search functionality to locate properties would be helpful, especially within nested levels.
        & Implemented in the toolbar.
        All findings are highlighted in the GUI panel. \\
        \midrule
        %& 4. The Schema editor GUI panel contains extensive metadata, but the code editor panel remains empty. & 4. bla bla bla \\
        %\midrule
        The GUI panel feels overwhelming to the user due to many variations in styling and color of the GUI elements.
        &
        We slightly reduced the number of different styling by no longer showing required properties in bold face and instead just show an asterisk next to it. \\
        \midrule
        The cursor should not have the clickable animation when hovering over non-clickable fields in the GUI editor.
        &
        Now we only show the clickable animation when hovering over clickable GUI components. \\
        \midrule
        In drop-down menus we do not need a button to clear the selection.
        &
        We disabled the option of clearing the selection. \\
        \midrule
        If the type of a property is ``any'', it should not be interpreted as the ``string'' type in the GUI panel.
        &
        We show a drop-down menu to the user, where they can select the type they want to use. \\
        \midrule
        Validation errors should not be highlighted via a warning symbol, but instead an error symbol.
        &
        We changed the warning symbol into an error symbol. \\
        \midrule
        After performing an undo or redo action, the cursor should jump to the corresponding location to reflect the changes made by the user.
        &
        Will be considered in future work. \\
        \bottomrule
    \end{tabular}
\end{table*}



\subsubsection{\textbf{RQ2:} Are users able to perform the followings types of tasks using the tool?}
Table~\ref{table:results} shows the accuracy and difficulties that the participants had when solving the tasks.
Accuracy is determined as the ratio of tasks that were solved correctly without the need of any hints to the total number of tasks.
With the help of hints, all tasks could be completed by every participant.


% TODO: Summarize results and answer the RQ (around one sentence for 2.1, 2.2, 2.3 each)

% Minye

\begin{table*}[!htbp]
    \centering
    \caption{Results of user study} \label{table:results}
    \begin{tabular}{lp{4.5cm}p{6cm}}
        \toprule
        &
        \thead{Accuracy \& Notes} & \thead{Difficulties} \\
        \midrule

        User Study 1 &
            Accuracy: 100\%(11/11)\newline
            Effective use of tooltips\newline
            Used both GUI and code editor

            &

            Task 3.2: Setting Vehicle Type to the highest level:\newline
            Took some time to find the field\\ \midrule

        User Study 2 &
            Accuracy: 91\%(10/11)\newline
            Used search functionality\newline
            Used both GUI and code editor

            &

            Task 4.2: Adding new property:\newline
            Could not find where to add new property in Schema Editor\newline
            Did not know how to edit property name \\ \midrule

        User Study 3 &
            Accuracy: 82\%(9/11)\newline
            Effective use of tooltips\newline
            Not familiar with JSON schema\newline
            Used only code editor in beginning and then only GUI editor after task 2.3.

            &

            Task 2.3: Duration of Simulation:\newline
            Did not think about using the GUI panel to retrieve information\newline
            Task 2.4: Validity of humidity:\newline
            Mistakenly thought humidity value was valid.\newline
            Did not consider red underline as an error\newline
            Task 4.2: Add new property:\newline
            Set property name in wrong place
         \\ \midrule

        User Study 4 &
            Accuracy: 82\%(9/11)\newline
            Solved tasks in a short time\newline
            Used only the GUI editor

            &

            Task 2.4: Validity of humidity:\newline
            Mistakenly thought humidity value was valid\newline
            Did not find out that he could use tooltips\newline
            Task 3.2: Setting Vehicle Type to the highest level: \newline
            Did not scroll down in Dropdown menu\newline
            Task 3.3: Validation Errors:\newline
            Thought red underline was spell checking \\ \midrule

        User Study 5 &
            Accuracy: 100\%(11/11)\newline
            Solved tasks in a short time\newline
            Used both GUI and code editor

            &

            Task 3.2: Setting Vehicle Type to the highest level: \newline
            Took some time to find the vehicle type \newline
            Task 4.2: Adding new property:\newline
        Did not set the property type at the beginning \\ \bottomrule
    \end{tabular}
\end{table*}



\begin{table*}

    \centering
    \caption{User Study 2 - Feedback and Resolution}
    \label{tab:user_study2}
    \begin{tabular}{p{0.45\linewidth}p{0.45\linewidth}}
        \toprule
        \thead{Feedback} & \thead{Resolution} \\
        \midrule
        A graph-based view would be more intuitive for handling complex data structures.
        &
        Will be considered in future work. \\
        \midrule
        Providing immediate feedback to users when they enter incorrect ranges is essential to prevent them from inputting invalid values into the property.
        &
        We now highlight schema violations by a red error symbol in the GUI panel and underlining the property name in red.
        Additionally, the tooltip lists all schema violations of a property. \\
        \midrule
        Validation errors should also be reflected in the GUI panel, including for child properties.
        &
        See point above.
        Also, now the tooltip lists schema violations of child properties. \\
        \midrule
        When dealing with an array, the display name of array elements (index) should be improved.
        Currently, the tool only shows just the element index.
        &
        We replaced the numerical labels by a standard programming notation, which is \texttt{propertyName[0]}, \texttt{propertyName[1]}, \ldots
        \\
        \midrule
        The input field next to the \textit{Add Item} button is confusing.
        Both the input field and the button can be used to create a new item, which is redundant.
        &
        We removed the input field next to the button. \\
        \midrule
        It would be more consistent, if all user input in the GUI panel was within the right column of the table.
        That in some scenarios user input is needed within the left column (for names of new properties) feels inconsistent.
        &
        Because of the nature of JSON schema, we retained the property name within the left column.
        To make it clear to the user that the property name can be edited, we added an \textit{edit} icon next to it. \\
        \midrule
        The search function for locating specific properties lacks clarity at first glance.
        It should provide an immediate response and extend to nested levels, rather than merely highlighting the higher-level findings.
        &
        The search now provides a list of results, and upon clicking on a particular result, it jumps to that result in the code panel and GUI panel.
        In the GUI panel, if the element is nested, its parents will be automatically expanded. \\
        \bottomrule
    \end{tabular}

\end{table*}

\begin{table*}
    \centering
    \caption{User Study 3 - Feedback and Resolution} \label{tab:user_study3}
    \begin{tabular}{p{0.45\linewidth}p{0.45\linewidth}}
        \toprule
        \thead{Feedback} & \thead{Resolution} \\
        \midrule
        Working with the schema editor is difficult for me.
        It does not feel intuitive.
        &
        We made the schema editor more intuitive by creating our own simplified JSON schema meta schema.
        For example, advanced JSON schema features are separated from the simple ones.
        See section~\ref{subsec:schema-editor} \\
        \bottomrule

    \end{tabular}

\end{table*}

\begin{table*}
    \centering
    \caption{User Study 4 - Feedback and Resolution} \label{tab:user_study4}
    \begin{tabular}{p{0.45\linewidth}p{0.45\linewidth}}
        \toprule
        \thead{Feedback} & \thead{Resolution} \\
        \midrule
        Modifying or renaming a new property in the GUI panel does not appear to take effect when double-clicking on it.
        &
        Renaming properties in the GUI panel can now be done using the \textit{edit} button next to the property name. \\
        \midrule
        When creating a new property in the schema editor, its sub-schema has to be selected, such as \textit{string property} or \textit{boolean property}.
        Additionally, the type of the property has to be selected by the user too.
        Therefore, for example, when creating a new \textit{string property}, the user has to select that it is a string two times.
        It would be much more intuitive if the selection needs to be done only one time.
        & We completely overhauled our JSON schema meta schema.
        Now, when creating a new property, the user will have to select the type only once. \\
        \midrule
        A toggle button should be implemented to enable and disable the code panel and GUI panel.
        & Only having a GUI panel or only having a code panel restricts the user unnecessarily.
        The interplay of both panels is what makes this tool most effective.
         If the user does not want to use one of the panels, they can resize that panel to a very small size.  \\
        \midrule
        When working with a particular property in the GUI panel the opacity of the other properties should be decreased, visually highlighting the property that currently is in focus. & Will be considered in future work. \\
        \midrule
        Simplify the schema editor to make it easier to work with, for those who are not very familiar with JSON schema.
        &
        Has been done, see section~\ref{subsec:schema-editor}. \\
        \bottomrule

    \end{tabular}

\end{table*}

\begin{table*}

    \centering
    \caption{User Study 5 - Feedback and Resolution} \label{tab:user_study5}
    \begin{tabular}{p{0.45\linewidth}p{0.45\linewidth}}
        \toprule
        \thead{Feedback} & \thead{Resolution} \\
        \midrule
        The search button is not immediately evident, making it challenging for users to locate the search function. &
        Instead of showing the search bar only when clicking the search button, we now always show it. \\
        \bottomrule

    \end{tabular}

\end{table*}


\subsubsection{\textbf{RQ3:} Would people use the tool in practice?} % Keyuri



% TODO: create table regarding "Would you use the tool in practice?" with one row per participant
% Keyuri




% maybe add more interpretations of result
