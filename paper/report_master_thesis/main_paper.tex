\documentclass[lettersize,journal]{IEEEtran}
\usepackage{amsmath,amsfonts}
\usepackage{algorithmic}
\usepackage{algorithm}
\usepackage{array}
%\usepackage{minted}
\usepackage[caption=false,font=normalsize,labelfont=sf,textfont=sf]{subfig}
\usepackage{textcomp}
\usepackage{lipsum}
\usepackage{xcolor}
\usepackage[export]{adjustbox}
\usepackage{stfloats}
\usepackage{url}
\usepackage{verbatim}
\usepackage{graphicx}
\usepackage{cite}
\usepackage[hidelinks]{hyperref}
\usepackage{enumitem}
\usepackage{listings}
\usepackage{booktabs}
\usepackage{caption}
\hyphenation{op-tical net-works semi-conduc-tor IEEE-Xplore} % todo remove
% updated with editorial comments 8/9/2021
\usepackage{utfsym}
\usepackage{makecell}
\renewcommand{\checkmark}{\usym{1F5F8}}
\newcommand{\cfgfiles}{configuration files}
\newcommand{\cfgfile}{configuration file}
\newcommand{\toolname}{\textit{MetaConfigurator}}
\newcommand{\jskeyword}[1]{\texttt{#1}} % use for json schema keywords

\colorlet{punct}{red!60!black}
\definecolor{background}{HTML}{EEEEEE}
\definecolor{delim}{RGB}{20,105,176}
\colorlet{numb}{magenta!60!black}

% adding code style for json
\lstdefinelanguage{json}{
 basicstyle=\small\ttfamily,
 numbers=left,
 xleftmargin=2.0ex,
 float=htb,
 numberstyle=\scriptsize,
 stepnumber=1,
 numbersep=8pt,
 showstringspaces=false,
 breaklines=true,
 frame=lines,
 backgroundcolor=\color{background},
 literate=
 *{0}{{{\color{numb}0}}}{1}
  {1}{{{\color{numb}1}}}{1}
  {2}{{{\color{numb}2}}}{1}
  {3}{{{\color{numb}3}}}{1}
  {4}{{{\color{numb}4}}}{1}
  {5}{{{\color{numb}5}}}{1}
  {6}{{{\color{numb}6}}}{1}
  {7}{{{\color{numb}7}}}{1}
  {8}{{{\color{numb}8}}}{1}
  {9}{{{\color{numb}9}}}{1}
  {:}{{{\color{punct}{:}}}}{1}
  {,}{{{\color{punct}{,}}}}{1}
  {\{}{{{\color{delim}{\{}}}}{1}
  {\}}{{{\color{delim}{\}}}}}{1}
  {[}{{{\color{delim}{[}}}}{1}
  {]}{{{\color{delim}{]}}}}{1},
  string=[s]{"}{"},
  comment=[l]{:\ "},
  morecomment=[l]{:"},
}

\captionsetup[lstlisting]{font=small, justification=justified, singlelinecheck=false}
\captionsetup[figure]{font=small, justification=justified, singlelinecheck=false}
\captionsetup[table]{font=sc, justification=centering, singlelinecheck=false, labelsep=newline}

\begin{document}

 \title{Integration of ontologies into MetaConfigurator and application in real world research communities}
 \author{Felix Neubauer}
%\author{IEEE Publication Technology,~\IEEEmembership{Staff,~IEEE,}
 % <-this % stops a space
%\thanks{This paper was produced by the IEEE Publication Technology Group. They are in Piscataway, NJ.}% <-this % stops a space
%\thanks{Manuscript received April 19, 2021; revised August 16, 2021.}}

% The paper headers
 %\markboth{Journal of best software 2023}%
 %{Shell \MakeLowercase{\textit{et al.}}: Revolutionizing Software}

%\IEEEpubid{0000--0000/00\$00.00~\copyright~2021 IEEE}
% Remember, if you use this you must call \IEEEpubidadjcol in the second
% column for its text to clear the IEEEpubid mark.

 \maketitle

 \begin{abstract}
 TODO
 \end{abstract}

 \begin{IEEEkeywords}
  JSON, YAML, configuration, schema, GUI
 \end{IEEEkeywords}


 \section{Introduction}\label{sec:introduction} %felix

Textual data formats, such as JSON, XML, and YAML, are widely used for structuring research data.
Manually editing data in these formats can be complex and time-consuming.
Graphical user interfaces (GUIs) can significantly reduce manual efforts and assist the user in editing the files.
However, developing a file-format-specific GUI requires substantial development and maintenance efforts.
We have addressed this challenge by the development of MetaConfigurator, an open-source web application that generates its GUI depending on a given schema.
It offers a unified view that combines the benefits of both GUIs and text editors, it enables schema editing within the same tool, and it supports advanced schema features, including conditions and constraints.
While a user study, which we conducted, suggests that MetaConfigurator is user-friendly and effective in enabling users to retrieve information from structured data files and in editing them, it also uncovered that the schema editor can be complicated for users not familiar with JSON schema.
Additionally, there are other known limitations, such as lack of complete YAML support. Furthermore, there are various possible features for which we believe they will significantly improve the user experience of MetaConfigurator and help it turn from an interesting idea into a tool that is applied in practice, in various domains in research.


% TODO
TODO

 \section{Related Work}\label{sec:research}
 % intro: felix

This section covers existing schema languages and existing approaches to generate UIs from them.
% We compare the schema languages briefly and discuss why JSON schema is the most useful language for our approach.
As our research is of a practical nature, we also consider gray literature such as specifications of schemas or websites.
\subsection{Schema Languages}\label{subsec:schemalanguages}

\textit{Schema languages} are formal languages that specify the structure, constraints, and relationships of data, for example in a database or structured data formats.

As this work is concerned with generating a GUI based on a schema, we need to choose a suitable schema language.
The following sections describe existing schema languages.
We will compare them in section~\ref{sec:evaluation-of-schema-languages} to determine which is the most suitable one for this work.

%paul
\subsubsection{JSON schema}

JSON is a common data-interchange format for exchanging data with web services, but also for storing documents in NoSQL databases, such as MongoDB\@\cite{marrs2017json}.
Because of the popularity of JSON, there is also a demand for a schema language for JSON\@.
One such language is JSON schema~\cite{jsonSchema, jsonschemaJSONSchema}.
Listing~\ref{lst:json-schema-example} shows an example of a JSON schema and listing~\ref{lst:json-example} shows an example of a JSON document that conforms to the schema.

JSON schema has evolved to being the de-facto standard schema language for JSON documents~\cite{baazizi2021empirical}.
Schemas for many popular \cfgfile{} types exist.
\textit{JSON schema store}\cite{schemastoreJSONSchema} is a website that provides over 600 JSON schema files for various use cases.
The supported file types include for example Docker compose or OpenAPI files.
%\cite{barbaglia, ChaeronySiffa2022} give further examples of JSON schema used in practice.

We remark that JSON schema and other schema languages for JSON can also be applied to YAML as JSON and YAML documents are of a similar structure (JSON is a subset of YAML).
Some syntactical details of YAML can, however, not be expressed with JSON schema.

% minye
\begin{lstlisting}[language=json,firstnumber=1,caption={JSON schema example},captionpos=b,label={lst:json-schema-example}]
{
  "$id": "https://example.com
  /person.schema.json",
  "$schema": "https://json-schema.org
  /draft/2020-12/schema",
  "title": "Person",
  "type": "object",
  "properties": {
    "firstName": {
      "type": "string",
      "description": "first name."
    },
    "lastName": {
      "type": "string",
      "description": "last name."
    },
    "age": {
      "description": "Age",
      "type": "integer",
      "minimum": 0
    }
  }
}
\end{lstlisting}


\begin{lstlisting}[language=json,firstnumber=1,caption={JSON example for the schema in listing}~\ref{lst:json-schema-example},captionpos=b,label={lst:json-example}]
{
  "firstName": "John",
  "lastName": "Doe",
  "age": 21
}
\end{lstlisting}

% felix
\subsubsection{XSD and DTD}
For XML the two de-facto standard schema languages are Document Type Definition (DTD)\cite{dtd_spec} and XML Schema Definition (XSD)\cite{xsd_spec}.
XSD is the newer and more expressive format and in large parts replaces and supersedes the more limited format DTD\cite{dtd_vs_xsd}.
It is recommended by W3C as a schema language for XML documents\cite{xsd_spec}.
Multiple other schema languages have been proposed and developed but are relatively unknown compared to XSD\cite{xml_schemas_1,xml_schemas_2}.

\subsubsection{Other schema languages} %Minye
We also consider the following schema languages:
\begin{enumerate}[label=(\alph*)]
    \item CUE (Configure, Unify, Execute)\cite{cuelang} is a data validation and configuration language, which can be used with various data formats, such as JSON and YAML (it is a superset of both).
    It has several use cases, especially in configuration and data validation.
    \item Apache Avro~\cite{Apache-Avro} is an open-source project that provides data serialization and data exchange services for Apache Hadoop.
    It uses a JSON-based schema language.
    \item JSON Type Definition (JTD)~\cite{rfc8927} is a schema language for JSON documents, which is significantly simpler than JSON schema.
    \item Type Schema~\cite{Kappestein_2023} is a schema language for JSON documents, similar to JSON Type Definition but using a different syntax.
    \item GraphQL schema language~\cite{graphQL} is a schema language for GraphQL APIs.
    \item Protocol Buffers~\cite{protobufProtocolBuffers} is a language for data serialization by Google.
\end{enumerate}

We do not consider any graphical modeling languages, such as UML or ER diagrams, as they are not text-based.
Although they can be converted to text-based formats, their main purpose is to model data structures and relationships between them.
We also do not consider any ontology languages, such as OWL or RDF Schema, as they are not intended for data validation but rather for knowledge representation.
Future work could investigate if such languages are also useful for our use case.
Finally, we do not consider any programming languages as schema languages.
Technically, programming languages can be used to define data structures and constraints, but they are not intended for this purpose, and it would be very challenging to generate a GUI from them.

\subsection{Existing Approaches}\label{subsec:existing-approaches}
% felix
Our work focuses on assisting users in creating and maintaining \cfgfiles{} so that they are valid and adhere to a predefined schema.

%There exist several approaches that attempt to make maintaining \cfgfiles easier for the user.
%Most of these approaches depend on the schema of the configuration file to be known.
There exist techniques to validate \cfgfiles{} against a schema~\cite{JSON_schema_vailidation,JSONValidation,baeldung_2023}.
Usually, schema validation is done only internally, e.g., by web services or libraries.
However, there exist also approaches that use the schema to assist the user in creating and maintaining \cfgfiles.
IDEs, such as Visual Studio Code or IntelliJ IDEA, can validate \cfgfiles{} against a schema and provide the user with error messages.
Those IDEs also provide other features, such as auto-completion, syntax highlighting, and tooltips.
However, they typically do not provide a graphical user interface (GUI) for editing the \cfgfiles{} based on the schema.


\subsubsection{Form generation}\label{subsubsec:schema-to-gui}
% minye and felix and paul

Related to our work are approaches that generate a GUI from a schema.
This section covers form generators, i.e., approaches that generate a web form from a schema.
Such forms can assist the user in a multitude of ways, such as by tooltips, auto-completion (Figure~\ref{fig:gui_advantage_autocomplete}) and dropdown menus (Figure~\ref{fig:gui_advantage_choiceselection}).
By inherently adhering to the schema structure (in most cases), editing data with such GUIs significantly reduces configuration mistakes caused by the user.
Users who are not very familiar with the configuration schema profit most from the GUI assistance, but even experienced users benefit from it.

\begin{figure}[htb]
    \centering
    \includegraphics[width=2.5in]{figures/gui_advantage_autocomplete}
    \caption{Auto-Completion}
    \label{fig:gui_advantage_autocomplete}
\end{figure}

\begin{figure}[htb]
    \centering
    \includegraphics[width=2.5in]{figures/gui_advantage_choiceselection}
    \caption{Choice Selection}
    \label{fig:gui_advantage_choiceselection}
\end{figure}

%paul
There exist various approaches that generate web forms from a schema, for different frontend frameworks, e.g.,
\textit{React JSON Schema Form}\cite{githubGitHubRjsfteamreactjsonschemaform},
\textit{Angular Schema Form}\cite{githubGitHubJsonschemaformangularschemaform},
\textit{Vue Form Generator}\cite{githubGitHubVuegeneratorsvueformgenerator},
\textit{JSON Forms}\cite{jsonformsMoreForms},
\textit{JSON Editor}\cite{jsoneditoronlineJSONEditor}, and
\textit{JSON Form}\cite{githubGitHubJsonformjsonform}.
Those approaches are all based on JSON schema and generate a form that can be filled out by the user and
the resulting JSON document is validated against the schema.
If the user enters invalid data, the form shows an error message.
The generated forms usually have a specific component for each type of data, e.g.\ a text field for strings or a number field for numbers,
similar to our approach.
Figure~\ref{fig:jsonforms} shows an example of a generated form using JSON Forms.

Those techniques, however, only provide the GUI for editing the data, but not a text-based editor.
A text-based editor is useful, especially for experienced users, who prefer to edit the data directly.
Also, these techniques do not provide a way to edit the schema itself, but only the data.
The most significant limitation of all except the last two of the given approaches is that they also require a ``UI schema'' in addition to the JSON schema, which is used to configure the generated form.
While these configurations can be used to customize the generated form, they also need to be created and maintained by the schema author.
Consequently, those approaches cannot be used to generate a GUI for any arbitrary schema, but manual effort is required to create the UI schema.

\begin{figure}[htb]
    \centering
    \includegraphics[width=2.5in]{figures/jsonforms}
    \caption{JSON Forms, example for a generated form}
    \label{fig:jsonforms}
\end{figure}

Adamant~\cite{siffa2022adamant} is a JSON-schema based form generator specifically designed for scientific data.
It generates a GUI from a JSON schema, allows editing and creating JSON schema documents
and differentiates between a schema edit mode and a data edit mode.
It supports a subset of JSON schema, which is sufficient for many use cases.
In addition to that, it supports the extraction of units from the description of a field, which is helpful for scientific data.
Figure~\ref{fig:adamant} shows an example in the schema edit mode.
%Adamant differs to \toolname{} in that it does not provide a text-based editor for the schema and that it is specifically designed for scientific data,
%while our approach is more general.
Limitations of Adamant are first, it does only support a subset of JSON schema, which is sufficient for many use cases, but not for any arbitrary schema.
Second, it does not provide a text-based editor for neither the schema nor the data.
Finally, it is specifically designed for scientific data, which makes it less suitable for other use cases, especially large and complex schemas.

\begin{figure}[htb]
    \centering
    \includegraphics[width=2.5in]{figures/adamant}
    \caption{Adamant, example for a form in edit mode}
    \label{fig:adamant}
\end{figure}


\subsubsection{Schema editors}\label{subsubsec:schema-editors}
In \toolname{} we aim to provide a GUI for both editing \cfgfiles{} and editing the schema.
For the latter, there exist several so-called schema editors, which are tools for creating and editing schemas that are
either text-based or graphical (or both).

\textit{JSON Editor Online}\cite{jsoneditoronlineJSONEditor} is a web-based editor for JSON schemas and JSON documents.
It divides the editor into two parts, where one part can be used to edit the schema and the other part can be used to edit a JSON document,
which is validated against the schema.
The editor provides various features, such as syntax highlighting and highlighting of validation errors (Figure~\ref{fig:jsoneditoronline}).
It provides a text-based or tree-based view for editing the JSON documents.
For simple objects that are not further nested, it also provides a table-based view (Figure~\ref{fig:jsoneditoronline_table}).
However, the features of the editor are very limited.
For example, it does not provide any assistance for the user, such as tooltips or auto-completion.
For new documents, it does not show the properties of the schema, so the user has to know the schema beforehand.

There also exists a variety of schema editors that are paid software, such as \textit{Altova XMLSpy}\cite{altovaEditorXMLSpy},
\textit{Liquid Studio}\cite{liquidtechnologiesJSONSchema}, \textit{XML ValidatorBuddy}\cite{xmlbuddyEditorValidator},
\textit{JSONBuddy}\cite{jsonbuddyJSONSchema}, \textit{XMLBlueprint}\cite{xmlblueprintEditorXMLBlueprint},
and \textit{Oxygen XML Editor}\cite{oxygenxmlCompleteSolution}.
Those are editors for XML or JSON schema, mostly with a combination of text-based and graphical views.
These tools are not web-based and not open-source.
Furthermore, they do not focus on editing a JSON document based on a schema,
but rather only on editing the schema itself.
\begin{figure}[htb]
    \centering
    \includegraphics[width=3in]{figures/jsoneditoronline}
    \caption{JSON Editor Online}
    \label{fig:jsoneditoronline}
\end{figure}
\begin{figure}[htb]
    \centering
    \includegraphics[width=3in]{figures/jsoneditoronline-table}
    \caption{JSON Editor Online, table view}
    \label{fig:jsoneditoronline_table}
\end{figure}


\subsubsection{Schema visualization}\label{subsubsec:schema-visualization}
Generating a GUI from a schema is related to schema visualization, for which several techniques exist~\cite{Frasincar2006, SILVA201928, 10.1145/1317353.1317362, 1173142}.
However, the focus of schema visualization is on providing a static visual representation of the schema
and not on providing a GUI for editing the schema.
Thus, we do not consider schema visualization approaches in this work.
However, future work can investigate how such techniques could be embedded in our approach.


\section{Evaluation of Schema Languages}\label{sec:evaluation-of-schema-languages}


We evaluate the schema languages mentioned in section~\ref{subsec:schemalanguages} to determine which is the most suitable one for this work.


\subsection{Evaluation Criteria}\label{subsec:evaluation-criteria} % evaluation by paul, some criteria ideas by felix

Ideally, the schema language of \toolname{} is both popular and supported by numerous tools and libraries as well as expressive enough to express the features we need.
We use the following criteria and metrics:
\begin{enumerate}
% metric: Stackoverflow question # with schema language as tag
    \item \textbf{Practical usage} --- Ideally our approach uses a schema language that is already known by many developers.
    As indicator of the practical usage we use the approximate search results on stackoverflow.com as metric.
    We acquire the results by querying the google search engine with the name of the schema language and ``site:stackoverflow.com'', which limits the search results to stackoverflow.com.
    This metric might also correlate with the complexity of the schema language as a more complex to use schema language will likely lead to more questions asked on the site.
    Nevertheless, we assume that a significantly higher number of results indicates that a language is more known than others.

    Additionally, we investigate how well the schema languages are supported by IDEs and code libraries:
    \begin{enumerate}
        \item \textit{Tool support} --- We used the 10 most popular IDEs\cite{mostpopularides} and checked if the IDE supports the schema language either natively or by a plugin.
        Here, support means that either the IDE is capable of validating documents against a schema in the schema language or supports creating schema files, e.g., by using syntax highlighting for the schema language.
        % # node modules with schema language keyword....
        \item \textit{Library support} --- As we implement a web-based tool, JavaScript or TypeScript based tools are helpful for our approach, e.g., so we can reuse a package for schema validation.
        We investigate the number of node modules existing that are related to the schema languages by querying the node module search on \url{www.npmjs.com} with the name of the schema language.

    \end{enumerate}

    % # of cases fulfilled from below
    \item \textbf{Expressiveness} --- We evaluate how expressive each of the schema languages is, i.e., what possible constructs the language is able to express.
    We define eight requirements on the language features that we consider helpful for our approach.
    The number of requirements a schema language fulfills is our metric that indicates how expressive the language is.
    Table~\ref{tab:comparison} reports the results.
    The nine requirements are:
    \begin{enumerate}
        \item \textit{Simple types} --- This is fulfilled if the schema language provides the possibility to define simple data types, at least strings, numeric types, and a boolean type.
        This is a fundamental feature for our approach.
        \item \textit{Complex types} --- This is fulfilled if the schema language provides the possibility to define complex data types, at least records and arrays.
        This is crucial feature for our approach as configuration files are often structured data rather than plain key-value pairs.
        \item \textit{Descriptions} --- This is fulfilled if the schema language provides the possibility to add descriptions to fields.
        This is helpful in a schema-to-GUI approach as the description can be shown to the user, providing potential helpful information on how a field should be filled.
        \item \textit{Examples} --- This is fulfilled if the schema language provides the possibility to add example values.
        This is helpful in our approach as the example values can serve as placeholders in the GUI editor.
        \item \textit{Default values} --- This is fulfilled if the schema language provides the possibility to add default values which are assumed in an absence of a value.
        This helpful information can be displayed to the user or used as placeholder values.
        \item \textit{Optional values} --- This is fulfilled if the schema language provides the possibility to declare values as optional or required.
        Often it is not necessary to provide all values in a configuration file, so it is helpful to mark fields as required or optional in the GUI editor.
        \item \textit{Constraints} --- This is fulfilled if the schema language provides the possibility to constrain values of fields, e.g., maximum length of strings.
        To be exact, for this evaluation we require that at least two of the following constraints can be expressed by the schema language:
        \begin{itemize}
            \item The length of strings can be limited.
            \item The range of numeric types can be limited, e.g., to only positive values.
            \item The valid values of a field can be restricted to a finite amount of values (enumeration).
            \item The format of a string field can be constrained to a certain pattern.
        \end{itemize}
        This is a helpful feature for our approach as often not all possible values are valid for specific fields in configuration files.
        \item \textit{Conditions} --- This is fulfilled if the schema language provides the possibility to define conditional dependencies between fields.
        This is an advanced feature that is helpful because it allows to express that a particular field must be given only if another field has a specific value.
        \item \textit{References} --- This is fulfilled if the schema language provides the possibility to define reusable sub-schemas that can be referenced in other parts of the schema.
        This is often useful in practice to reuse common data structures.

    \end{enumerate}

    %\item \textbf{Support for XML, YAML, or JSON} ---

    % For our approach we aim to use the same or at least a very similar editor for both editing the actual \cfgfiles and the schema files. Consequently, the schema language should be a subset of the language that the \cfgfiles are written in, i.e. the document language. For example JSON schema files are JSON files and JSON schema is used to validate JSON files, so here this criteria is fulfilled. In constrast, DTD is a schema language for validating XML files but a DTD file is not a valid XML file.

\end{enumerate}


\begin{table*}[]
    \centering
    \caption{Evaluation of different schema languages\label{tab:all}}
    \begin{tabular}{@{}lrrrr@{}}
        \toprule
        \textbf{Schema language} &
        \textbf{\# Search results } &
        \textbf{IDE support} &
        \textbf{\# Node packages} &
        \textbf{Expressiveness} \\ \midrule
        JSON schema &
        245,000 &
        8 / 10 &
        4,536 & 9 /9 \\
        XSD & 151,000 & 8 / 10 & 116 & 8 / 9 \\
        DTD & 69,700 & 9 / 10 & 34 & 6 / 9 \\
        CUE & 10,500 & 4 / 10 & 97 &   8 / 9 \\
        Avro & 20,000 & 8 / 10 & 211 &  5 / 9 \\
        JSON Type Definition (JTD) &  109 & 0 / 10 &  17 & 5 / 9 \\
        TypeSchema &  8,450 & 0 / 10 & 5 & 8 / 9 \\
        Protobuf &  44,800 & 9 / 10 & 1,210 & 4 / 9 \\
        GraphQl schema & 31,000 & 7 / 10 & 1,509 & 6 / 9\\ \bottomrule
    \end{tabular}
\end{table*}
%\label{fig:my_label}
%\end{figure}

% IDE                | JSON schema | XSD | DTD | CUE | Avro | protobuf | GraphQL schema |
% Visual Studio      | yes         | yes | yes | x   | yes  | yes      | yes            |
% Visual Studio Code | yes         | yes | yes | yes | yes  | yes      | yes            |
% Eclipse            | yes         | yes | yes | x   | yes  | yes      | x              |
% pyCharm            | yes         | yes | yes | yes | yes  | yes      | yes            |
% Android Studio     | yes         | yes | yes | yes | yes  | yes      | yes            |
% IntelliJ           | yes         | yes | yes | yes | yes  | yes      | yes            |
% NetBeans           | x           | yes | yes | x   | x    | yes      | x              |
% RStudio            | x           | x   | x   | x   | x    | x        | x              |
% Atom               | yes         | x   | yes | x   | yes  | yes      | yes            |
% Sumblime Text      | yes         | yes | yes | x   | yes  | yes      | yes            |
% ======================================================================================|
%                    | 8 / 10      | 8   | 9   | 4   | 8    | 9        | 7              |

\begin{table*}
    \centering
    \caption{Comparison of expressiveness of different schema languages
    \label{tab:comparison}}
    \begin{tabular}{@{}llllllllllr@{}}
        \toprule
        \textbf{Schema language} &
        \thead{Simple \\ types} &
        \thead{Complex \\ types} &
        \thead{Descriptions} &
        \thead{Example \\ values} &
        \thead{Default \\ values} &
        \thead{Optional \\ values} &
        \thead{Constraints} &
        \thead{Conditions}  &
        \thead{References}  &
        \thead{Result}\\ \midrule
        JSON schema &
        \checkmark &
        \checkmark &
        \checkmark &
        \checkmark &
        \checkmark &
        \checkmark &
        \checkmark &
        \checkmark &
        \checkmark & 9 / 9\\
        XSD &
        \checkmark &
        \checkmark &
        \checkmark &
        x &
        \checkmark &
        \checkmark &
        \checkmark &
        \checkmark &
        \checkmark & 8 / 9 \\
        DTD &
        \checkmark &
        \checkmark &
        x &
        x &
        \checkmark &
        \checkmark &
        x &
        \checkmark &
        \checkmark & 6 / 9\\
        CUE &
        \checkmark &
        \checkmark &
        \checkmark &
        \checkmark &
        x &
        \checkmark &
        \checkmark &
        \checkmark &
        \checkmark & 8 / 9\\
        Avro &
        \checkmark &
        \checkmark &
        x &
        x &
        \checkmark &
        \checkmark &
        x &
        x &
        \checkmark & 5 / 9\\
        JTD &
        \checkmark &
        \checkmark &
        x &
        x &
        x &
        \checkmark &
        x &
        \checkmark &
        \checkmark & 5 / 9\\
        TypeSchema &
        \checkmark &
        \checkmark &
        \checkmark &
        x &
        \checkmark &
        \checkmark &
        \checkmark &
        \checkmark &
        \checkmark & 8 / 9\\
        protobuf &
        \checkmark &
        \checkmark &
        x &
        x &
        x &
        \checkmark &
        x &
        x &
        \checkmark & 4 / 9 \\
        GraphQL schema &
        \checkmark &
        \checkmark &
        \checkmark &
        x &
        \checkmark &
        \checkmark &
        x &
        x &
        \checkmark & 6 / 9\\
        \bottomrule
    \end{tabular}
\end{table*}

% paul
\subsection{Evaluation results}\label{subsec:evaluation-results}

Tables~\ref{tab:all} and~\ref{tab:comparison} show the results of our evaluation.
We come to the conclusion that JSON schema is sufficiently popular and expressive that we choose to use it as the schema language for our approach.
The other schema languages are either less expressive or less popular.
This result is in line with the work of Baazizi et al.~\cite{baazizi2021empirical}, who also found over 80,000 JSON schema files on GitHub,
and with their claim that JSON schema is the de-facto standard for JSON schema languages.

% paul

%\subsection{Schema inference}\label{subsec:schema-inference}
%
%Schema inference is the process of deriving a schema from existing data.
%For our use case, this means inferring JSON schema from JSON documents.
%Frozza et al.~\cite{8424731} and Klettke et al.~\cite{klettke} present algorithms for JSON schema inference from JSON data of NoSQL data storages.
%Baazizi et al.\cite{Baazizi2019} also investigate schema inference from massive data sets but their approach uses its own type system rather than JSON schema.
%
%In our tool we only aim to infer a schema from a single sample, as an optional assisting feature for our users, for which various libraries exist~\cite{githubGitHubJsonsystemspublic, githubGitHubSaasquatchjsonschemainferrer, probst_siegel_2023}.



 \section{Making \toolname{} more user friendly}\label{sec:design_and_implementation}
 

 \subsection{Architecture improvements}\label{subsec:architecture_rework}
 The architecture of \toolname{}, as discussed and implemented in \cite{metaconfigurator} offers enough modularity to support different data formats and implement different \textit{panels}, such as the \textit{text editor panel} or the \textit{GUI editor panel}. 
However, it also is lacking in multiple aspects.
In this section, those limitations are addressed and architectural improvements suggested and implemented.

\subsubsection{Managing multiple schemas and data files simultaneously}
The original design features a single source of truth store, which is responsible for the currently loaded schema and \cfgfile{}. 
There can only be one schema and \cfgfile{} loaded at any given time.
When the user changes the tool \textit{mode} $\in$ \{\textit{File editor}, \textit{Schema editor}, \textit{Settings}\}, the schema and \cfgfile{} in the store are replaced accordingly.
Table \ref{tab:schema_and_file_data_by_mode} illustrates which file data and schema are used for the different modes.
Most subcomponents of the tool access the central store directly, instead of receiving the current schema and file as a parameter, that is passed through the component hierarchy. 
This comes with one major drawback: it is not possible to have the schema and data of different views loaded at the same time.
However, being able to simultaneously show content of different modes could significantly help the user.
For example, in the schema editor view, the file editor GUI could be shown too, giving the user a preview of the GUI that their schema will result in.
\begin{table}[!t]
\caption{File data and schema for the different modes}
\label{tab:schema_and_file_data_by_mode}
\centering
\begin{tabular}{lll}
\toprule
\textbf{Mode} & \textbf{Effective File Data} & \textbf{Effective Schema} \\
\midrule
File editor   & User data                    & User schema               \\
Schema editor & User schema                  & JSON Schema Meta Schema          \\
Settings      & Settings data                & Settings schema           \\
\bottomrule
\end{tabular}
\end{table}

To get rid of this limitation, we remove the $sessionStore = \{currentSchema, currentFileData, ...\}$.
Instead, we create a new store, which for each \textit{mode} stores the corresponding \textit{schema}, \textit{file data}, \textit{user selections} (e.g., which path is currently selected and which properties in the tree are expanded) and \textit{validation results}.
Additionally, we introduce mapping functions (e.g., $getDataForMode: mode \rightarrow {fileData}$), with which the content can be accessed.
Instead of the subcomponents of \toolname{} always accessing the one current schema and file data, they now each receive a \textit{mode} in their constructor and access the corresponding data.
For example, we can instantiate a text editor panel with the mode \textit{file editor}.
This panel will now use schema and data of the \textit{file editor} mode.
This architecture allows us to arrange panels of different types and different modes in an arbitrary manner, which brings us to our next point.



\subsubsection{Dynamic panel arrangement}\label{subsubsec:dynamic_panels}
The \textit{GUI editor} (todo: reference) or \textit{file editor} (todo: reference) are what we call \textit{panels}.
They have access to the \textit{schema} and \textit{file data} and visualize them in a particular manner.
They can also update the \textit{file data} or the \textit{currently selected element}, which are synchronized across all panels.
\toolname{} can be extended by more panels.
Table \ref{tab:panels_by_mode} shows which panels are used for the different tool modes, and their assigned panel modes (which data and schema to use).
For every mode, \toolname{} provides a text editor panel and a GUI editor panel, both using the data and schema of the corresponding mode.

\begin{table}[!t]
\caption{Panels and their data sources for the different modes}
\label{tab:panels_by_mode}
\centering\scriptsize
\begin{tabular}{@{}lll@{}}
\toprule
\textbf{Tool Mode} & \textbf{Panel 1: type(mode)} & \textbf{Panel 2: type(mode)} \\ \midrule
File editor        & Text editor(file editor)     & GUI editor(file editor)      \\
Schema editor      & Text editor(schema editor)   & GUI editor(schema editor)    \\
Settings           & Text editor(settings)        & GUI editor(settings)         \\ \bottomrule
\end{tabular}
\end{table}

To allow for more variation in the panels configuration (arrangement), we turn it into a setting.
This setting can be adjusted by the user (see figure \ref{fig:panels_definition_file_editor}) or by \toolname{} itself.
Table \ref{tab:panels_by_mode_schema_editor_with_preview} shows an alternative configuration for the panels of the schema editor.
It contains an additional GUI editor panel, which uses the data and schema from file editor mode. 
This configuration results in a schema editor, where the user gets a (functional) preview of the GUI that their schema will generate, see figure \ref{fig:panels_for_modified_schema_editor_config}.


\begin{figure}[!t]
    \centering
    \includegraphics[width=\columnwidth]{figures/panels_definition_file_editor}
    \caption{Settings GUI editor view with panels configuration for \textit{file editor} mode}
    \label{fig:panels_definition_file_editor}
\end{figure}


\begin{table}[!t]
\caption{Modified schema editor panel configuration}
\label{tab:panels_by_mode_schema_editor_with_preview}
\centering\tiny
\begin{tabular}{@{}llll@{}}
\toprule
\textbf{Tool Mode} & \textbf{Panel 1: type(mode)} & \textbf{Panel 2: type(mode)}& \textbf{Panel 3: type(mode)} \\ \midrule
Schema editor      & Text editor(schema editor)   & GUI editor(schema editor)   & GUI editor(file editor)    
  \\ \bottomrule
\end{tabular}
\end{table}


\begin{figure*}
    \includegraphics[width=\textwidth]{figures/panels_for_modified_schema_editor_config}
    \caption{Schema editor with a preview of the resulting GUI on the right}
    \label{fig:panels_for_modified_schema_editor_config}
\end{figure*}
%TODO: maybe screenshot of two text editor panels, one YAML, one JSON. And then put the schema editor preview in later section?


%\subsubsection{Decoupling}
%Having a single source of truth store that can be accessed from anywhere can be tempting.
%However, a component that uses the store depends on the store too.
%Sometimes, this dependency is not required and not even desired either.
%For example, the JSON schema preprocessing function is called with a JSON schema or sub-schema that is supposed to be preprocessed (e.g., references are resolved in a lazy manner).
%For reference resolving, the function requires all definitions of the properties to be resolved. 
%When only a sub-schema is provided to the function (this is the case when an object is expanded in the GUI editor), this sub-schema does usually not contain the definitions of the referenced data structures.
%To deal with this, \toolname{} accesses the store to receive the current schema, based on which it resolves the references.
%This, however, introduces the limitation that the function can only preprocess schemas which are currently loaded in the store and assigned to one of the modes.
%We decouple the function from the store, by introducing a second function parameter \textit{rootSchema}, which expects the root schema to use. 
%As a result, the function can take in any sub-schema and root schema, regardless of whether they are loaded in the store or not.

% TODO: not true, it is actually not implemented yet
 
 %\subsection{Making the code easier to understand}\label{subsec:maintainability}
 %TODO:

JsonSchema -> JsonSchemaWrapper %TODO: maybe rename to LazyResolvedJsonSchema
JsonSchemaType vs JsonSchemaTypePreprocessed
preprocessor -> lazy resolver
 
 
 \subsection{Enabling users to export their own \textit{configurators}}\label{subsec:custom_configurator}
 Developers and researchers can use \toolname{} to edit or create schemas and to edit data based on a given schema, in an assisted manner.
We intend for \toolname{} to also be a tool, which can be used to create a \textit{configurator} for any particular use case.
This use-case specific configurator can then be shared with anyone, by sharing a web URL with them.
When a user accesses the URL, \toolname{} is opened, with a pre-defined \textit{schema}, \textit{data file} and \textit{settings} loaded.
For example, a team of researchers might define a schema for experimental data. 
They could then generate a URL which will open \toolname{} with the experiment schema, an initial data file of their choice and settings of their choice.
This URL can then be shared with others researchers, who can use the tool to fill in their experimental data into the form (the GUI editor panel).
The benefit is that the users do not have to select a schema nor have to adjust any settings, but instead they immediately receive a fully functional \textit{configurator} for their use-case.
Simultaneously, the text editor panel would show and teach the researchers how the actual JSON or YAML data looks like, when adhering to the schema.
Apart from the field of research, this approach can be applied for any kind of structured data and a corresponding domain-specific \textit{configurator} URL generated.

\subsubsection{Basic support for pre-defined data}\label{subsubsec:custom_configurator_basic}
We implement support for loading pre-defined data based on the \textit{query string} in the URL.
Therefore, we introduce the following optional query string parameters:
\begin{itemize}
	\item \textbf{data:} URL of the data file to load
	\item \textbf{schema:} URL of the schema file to load
	\item \textbf{settings:} URL of the settings file to load
\end{itemize}

Additionally, we introduce a new settings property \textit{toolbarTitle}, which defines the title that is displayed on the \toolname{} toolbar on top of the page.

Figure \ref{fig:custom_configurator} shows the resulting view of \toolname{}, when opening it using the exemplary URL \underline{logende.org/meta-configurator/?data=a\&schema=b\&settings=c}, with \textit{a}, \textit{b}, \textit{c} being URLs to a data, a schema and a settings file.
\toolname{} is opened in the \textit{file editor} mode, with a pre-defined \textit{Self-Driving Vehicle} schema, data file and settings.
The settings file defines the custom title "Autonomous Vehicle Editor", which we can see in the top toolbar in the figure.
Notice that the user will not see the usual schema selection dialog and will not have empty files, but instead start with the pre-defined data.


\begin{figure*}
    \includegraphics[width=\textwidth]{figures/custom_configurator}
    \caption{\toolname{} with a pre-defined \textit{data}, \textit{schema} and \textit{settings} file. The settings file has the property \textit{toolbarTitle} set as "Autonomous Vehicle Editor".}
    \label{fig:custom_configurator}
\end{figure*}


\subsubsection{Advanced support for exporting custom configurators}

With the features described in section \ref{subsubsec:custom_configurator_basic} it is possible for anyone to hand-craft a \toolname{} URL that will load pre-defined data, if the corresponding files are already stored and accessible somewhere.
However, it has the following limitations:
\begin{enumerate}
	\item The user has to store the raw files somewhere, in an accessible manner
	\item The user has to hand-craft the \toolname{} URL with the query string
	\item The resulting URL is very long
\end{enumerate}

To make the process easier for the user, we can add a button in the \toolname{} UI, which will perform the following steps:
\begin{enumerate}
	\item Store user files in backend
	\item Generate URL that refers to the files in the backend
	\item Shorten the URL and store it in the backend too
	\item Return the shortened URL to the user
\end{enumerate}

This way, anyone can generate their own "configurator URL" with the click of a button.

TODO
 
 
 \subsection{Simplified and customizable schema editor}\label{subsec:better_schema_editor}
 Our previously conducted user study suggested that the \textit{schema editor} can be complicated for the user, especially if they have never worked with JSON schema before\cite{metaconfigurator}.


\subsubsection{Making the schema editor more simple}

The reason for the complexity of the schema editor lies in the expressiveness of JSON schema and because \toolname{} supports most of its keywords and features.
This is illustrated by figures \ref{fig:schema_editor_choose_jsonschema}-\ref{fig:schema_editor_choose_type}: when creating a new property for an object, JSON schema allows for the property to be a \textit{boolean schema} (\texttt{true} or \texttt{false}) or a \textit{sub-schema}.
For every \textit{sub-schema}, the \texttt{type} can either be empty, an array of types or a singular type.
All those options that JSON schema offers result in options that the user of the schema editor has to navigate through.

\begin{figure}[!t]
    \centering
    \includegraphics[width=\columnwidth]{figures/schema_editor_choose_jsonschema}
    \caption{Drop-down menu when defining a new property: selection of either a boolean schema or a sub-schema}
    \label{fig:schema_editor_choose_jsonschema}
\end{figure}



\begin{figure}[!t]
    \centering
    \includegraphics[width=\columnwidth]{figures/schema_editor_choose_array_or_single_type}
    \caption{Drop-down menu when defining a new sub-schema: selection of either a single type or a type union}
    \label{fig:schema_editor_choose_array_or_single_type}
\end{figure}



\begin{figure}[!t]
    \centering
    \includegraphics[width=\columnwidth]{figures/schema_editor_choose_type}
    \caption{Drop-down menu when defining a single type}
    \label{fig:schema_editor_choose_type}
\end{figure}


Other schema editors, such as Adamant \cite{todo} and TODO are easier to use, because they are simpler and do not support as many keywords and options as \toolname{}. 
In practice, many JSON schema keywords are used infrequently\cite{baazizi2021empirical} and the full expressiveness of JSON schema is not needed.
To allow for an easier schema editing experience, while maintaining the expressiveness and support for most JSON schema keywords, we introduce JSON meta schema parameters.
The user can decide themselves which advanced JSON meta schema features they want to use, and where they want to work with the simplified meta schema.
We introduce the following boolean parameters:
\begin{itemize}
	\item \textbf{allowBooleanSchema:} Whether a JSON Schema definition can also be just \texttt{true} or \texttt{false}. Having this option enabled will increase the choices that have to be made when defining a sub-schema in the schema editor.
	\item \textbf{allowMultipleTypes:} Whether an object property can be assigned to multiple types (e.g., \texttt{string} and \texttt{number}). Having this option enabled will increase the choices that have to be made when defining the type of a sub-schema in the schema editor, but also allows more flexibility. An alternative to defining multiple types directly is using the \texttt{anyOf} or \texttt{oneOf} keywords.
	\item \textbf{showAdditionalPropertiesButton:} Most schemas allow additional properties (e.g., adding properties to the data that are not defined in the schema). To support this in the schema editor, it would always provide an \textit{Add Property} button to allow adding properties unknown to the schema. In practice, this option is not used much, but it can confuse the user. For example, they might try adding new fields for their schema by using this button, although that does not have any effect on the schema.
	\item \textbf{objectTypesComfort:} This is a comfort feature: the original JSON Meta Schema allows properties of a particular type to have example values, constant values, default values or enum values of different types. For example, a field for numbers could have a string as a default value. This meta schema option forces the same type for all these values. This enables the tool to auto-select the corresponding type in the schema editor, avoiding the need for the user to manually select the types. 
	
	Warning: due to incompatibility, this option will disable schema editor support for defining the items of an array, as well as support for many advanced keywords, such as conditionals and \texttt{not}.
\end{itemize}
The customizability of the schema editor is achieved by a component, which we call \textit{meta schema builder}.
Based on the meta schema parameters, it builds a custom JSON meta schema, which is then used to create the schema editor GUI.
%It starts with our original (advanced) JSON meta schema, and based on the parameters, applies transformations on this meta schema, to simplify it.
We add a button to the schema editor (figure \ref{fig:schema_editor_advanced_mode_button}), which allows the user to toggle between an advanced meta schema (boolean schema and multiple types allowed, no object types comfort feature and showing the button to add additional properties) and a simple one (opposite configuration).
Furthermore, in the settings menu, the user can set the values for the individual parameters.
Figure \ref{fig:schema_editor_simple_choose_type} shows the drop-down menu for defining the type of a property, when using the simplified meta schema.
The type can be set directly.
Compare this with the advanced meta schema, where first the user had to select \texttt{sub-schema}, then \texttt{single type} and only then could select the actual type.

\begin{figure}[!t]
    \centering
    \includegraphics[width=0.5\columnwidth]{figures/schema_editor_advanced_mode_button}
    \caption{Button to enable the advanced schema editor, by using a more advanced meta schema}
    \label{fig:schema_editor_advanced_mode_button}
\end{figure}

\begin{figure}[!t]
    \centering
    \includegraphics[width=\columnwidth]{figures/schema_editor_simple_choose_type}
    \caption{Drop-down menu when defining the type of a new property}
    \label{fig:schema_editor_simple_choose_type}
\end{figure}



\subsubsection{Providing a (functional) preview of the resulting GUI}

The file editor GUI, which is created from the user schema, helps the user understand their schema.
It shows the properties defined in the schema, the titles and descriptions in the schema, whether a property is \texttt{required} or \texttt{deprecated}, and more.
To access the file editor, the user has to navigate from the schema editor view to the file editor view.
They cannot view both the schema editor and the resulting GUI simultaneously.
However, the improved architecture, as described in section \ref{subsubsec:dynamic_panels}, does make different panel arrangements possible.
We make use of that architecture and add a button to the schema editor (figure \ref{fig:schema_editor_show_preview}), which allows the user to toggle between showing the resulting file editor GUI in a third panel and hiding it. 


\begin{figure}[!t]
    \centering
    \includegraphics[width=0.5\columnwidth]{figures/schema_editor_show_preview}
    \caption{Button to show the file editor GUI in a third panel}
    \label{fig:schema_editor_show_preview}
\end{figure}



 
 \section{Graph View Panel}\label{sec:graph_view}
 \input{graph_view}
 
 

 \section{Ontology Integration}\label{sec:ontology_integration}
 
\subsubsection{User Interface}\label{subsec:overview}
The design of \toolname{} is inspired by another tool\cite{githubBspEditor} by one of the authors, which is a GUI program that assists users in editing configuration files of BossShopPro\cite{bossshoppro}.
That tool provides the user a code panel for editing configuration files of that domain in a text editor, as well as a GUI panel, where the user can edit their configuration file using GUI components.
\toolname{} differs from that tool by being generic, instead of being bound to a certain domain, by having a much more expressive schema language, a schema editor, and many other features that improve the user experience, such as a search functionality.
%TODO: maybe move to related work?

Before we dive into the architecture and detailed design of \toolname{}, this section provides an overview from the view of the user.

The user interface has three distinct views:
\begin{enumerate}
	\item File editor (figure \ref{fig:fileeditor}): In this view, the user can modify their \cfgfile{}, based on a schema.
	\item Schema editor (figure \ref{fig:schemaeditor}): In this view, the user can modify their schema.
	\item Settings (figure \ref{fig:settings} in appendix): In this view, the user can adjust parameters of the tool.
\end{enumerate}


\begin{figure*}
    \includegraphics[width=\textwidth]{figures/fileeditor}
    \caption{UI of file editor view. Different components highlighted in red: 1) button to switch to other view (e.g. to Schema Editor view), 2) Toolbar with various functionality, 3) Code panel, 4) GUI panel}
    \label{fig:fileeditor}
\end{figure*}

\begin{figure*}
    \includegraphics[width=\textwidth]{figures/schemaeditor}
    \caption{UI of schema editor view}
    \label{fig:schemaeditor}
\end{figure*}


 \section{Application in real research communities}\label{sec:application}
 \input{application}



 \section{Discussion}\label{sec:discussion}
 This section discusses the implications of our work and future work.


\subsection{Implications of our Work}\label{subsec:implications-of-our-work}

\toolname{} provides a novel approach for editing \cfgfiles{}
by combining the advantages of a GUI and a code editor.
Our user study suggests that \toolname{} allows users to successfully modify files within the constraints of a schema and to modify the schema itself.
The participants rated our tool as intuitive to use and responded that they would use it themselves.

However, the user study also revealed some limitations of \toolname{}.
Editing schemas with \toolname{} is not as intuitive as editing \cfgfiles{},
especially for users who are not familiar with JSON schema.
JSON schema may be feature-rich and expressive, but it is also complex and
hard to understand for new users.
The next section discusses how future work might address these limitations.
%Specialized editors for schemas, such as those discussed in section~\ref{subsubsec:schema-editors},
%may be more suitable for this task.

\subsection{Future Work}\label{subsec:future-work}
To make \toolname{} more useful for users who are not familiar with JSON schema,
there are several possible improvements.
First, a visual schema editor could be added to \toolname{}, similar to schema editors
discussed in section~\ref{subsubsec:schema-editors}.
These would provide a graph view of the schema, which is easier to understand than
the JSON schema in text form or the tree view in \toolname{}.
Second, \toolname{} could provide more guidance for users who are not familiar with JSON schema,
e.g., by providing an interactive tutorial or supporting a less complex schema language.

There are many other possible improvements to \toolname{}.
A desktop version of \toolname{} could be developed, which would allow users to
edit files on their local machine, which is more convenient than loading them into the web application.
Similarly, integration into other tools, such as IDEs, could be helpful for many users.
\toolname{} currently only supports JSON schema draft 2020--12, but it could be extended to support
other drafts by converting imported schemas to the latest draft.
Furthermore, YAML is not fully supported yet, which could be added in the future.
Another point that can be addressed is the loss of formatting and comments in YAML documents when they are updated with new data.
This could be avoided by replacing only the section in the YAML document that corresponds to the change, instead of replacing the complete document.
To allow for different styles of formatting, the user could be provided with global formatting style settings (such as level of indentation or whether in YAML strings should be in quotation marks or not).
To deal with the loss of comments, a technique that keeps track of any comments in the text and then restores them after the text is replaced could be implemented. This has already been done in another tool of one of the authors\cite{githubBspEditor}.

Finally, \toolname{} could be extended to support code generation, e.g., for generating
Java classes from a JSON schema, which is useful for developers.

To improve the user study, it could be repeated with more participants,
so that the results are more representative.
Instead of just having participants solve tasks, it could also be interesting to
have one group of participants solve tasks with \toolname{} and another group solve
the same tasks with only a text editor.
This way, we could evaluate whether \toolname{} is more efficient than
just using a text editor.




 \section{Conclusion}\label{sec:conclusion} % Felix

Conclusion here



\section*{Acknowledgments}
I want to thank my supervisors, Prof.\ Dr.\ Jürgen Pleiss and Prof.\ Dr.\ Benjamin Uekermann, for their support and guidance throughout this project.
% TODO: others, such as Jan Range

%\clearpage

 \bibliographystyle{IEEEtran}
 \bibliography{literature}


\appendices\section{User Study}\label{sec:user-study}  % Minye
\subsection{Interview Tasks}\label{subsec:tasks}
This part is about the newest version of interview tasks we prepared to conduct our user study.\\
\textbf{Remark:} Task 3 was added after the first user study and also some details in other tasks were modified.
\subsubsection{Introduction}
For these tasks you are presented a schema that you have not seen or worked with before.
We have prepared a schema of a made-up simulation software in which self-driving cars are simulated.
There is one self-driving car that has to navigate from a start point to an end point but there are also other vehicles and pedestrians simulated.
For these tasks, the exact details are not important.

\subsubsection{Task 1: Setup}
\begin{enumerate}
    \item Go to https://paulbredl.github.io/config-assistant/
    \item Select the option ``Select a Schema'': ``Example schema'', then ``Autonomous Vehicle Schema".
    \item Open the example configuration file we have sent to you. \\
          The following tasks will assume this schema and this example file.
\end{enumerate}

\subsubsection{Task 2: Questions}
\begin{enumerate}
    \item What is the name of the simulation?
    \item What is the weather in our simulated environment?
    \item What is the total duration of the simulation?
    \item Humidity is a subproperty of the Environment.
    Would 150 be a valid value for Humidity?
\end{enumerate}

\subsubsection{Task 3: Modifying the configuration file}
\begin{enumerate}
    \item Change the name of the simulation to Sim\_Advanced05.
    \item The VehicleType of the self-driving car is currently ``Level 3''.
    Change it to the highest possible level.
    \item The configuration file has validation errors, i.e., it is not valid according to the schema.
    \\ Find out what the errors are.
    Edit the file so it becomes a valid configuration.

\end{enumerate}

\subsubsection{Task 4: Modifying the schema}
\begin{enumerate}
    \item For many applications it is good practice or even required that a schema has a unique identifier, which usually is a URL.
          \\Set the \$id field of this schema to : ``https://www.example.com/self-driving-vehicle''.
    \item The simulation software will get a premium version in the next update, for which the user needs a license.
          The license key should be stored in the configuration file.
          Add a new property ``LicenseKey'' to the ``properties'' object.
          It should be of the type ``string''.
          The length of the key is at most 20 characters long.
    Add a short exemplary description.
    \item Go back to the file editor and verify that the new property ``LicenseKey'' is displayed and that the correct information is displayed when hovering over the property.
\end{enumerate}




%
% \vspace{11pt}
%
% \textbf{If you include a photo:}\vspace{-33pt}
%\begin{IEEEbiography}%[{\includegraphics[width=1in,height=1.25in,clip,keepaspectratio]{fig1}}]{Michael %Shell}
%Use $\backslash${\tt{begin\{IEEEbiography\}}} and then for the 1st argument use %$\backslash${\tt{includegraphics}} to declare and link the author photo.
%Use the author name as the 3rd argument followed by the biography text.
%\end{IEEEbiography}

%
% \textbf{If you will not include a photo:}\vspace{-33pt}
% \begin{IEEEbiographynophoto}{John Doe}
%  Use $\backslash${\texttt{begin\{IEEEbiographynophoto\}}} and the author name as the argument followed by the biography text.
% \end{IEEEbiographynophoto}


\end{document}


