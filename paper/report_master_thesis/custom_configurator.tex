Developers and researchers can use \toolname{} to edit or create schemas and to edit data based on a given schema, in an assisted manner.
We intend for \toolname{} to also be a tool, which can be used to create a \textit{configurator} for any particular use case.
This use-case specific configurator can then be shared with anyone, by sharing a web URL with them.
When a user accesses the URL, \toolname{} is opened, with a pre-defined \textit{schema}, \textit{data file} and \textit{settings} loaded.
For example, a team of researchers might define a schema for experimental data. 
They could then generate a URL which will open \toolname{} with the experiment schema, an initial data file of their choice and settings of their choice.
This URL can then be shared with others researchers, who can use the tool to fill in their experimental data into the form (the GUI editor panel).
The benefit is that the users do not have to select a schema nor have to adjust any settings, but instead they immediately receive a fully functional \textit{configurator} for their use-case.
Simultaneously, the text editor panel would show and teach the researchers how the actual JSON or YAML data looks like, when adhering to the schema.
Apart from the field of research, this approach can be applied for any kind of structured data and a corresponding domain-specific \textit{configurator} URL generated.

\subsubsection{Basic support for pre-defined data}\label{subsubsec:custom_configurator_basic}
We implement support for loading pre-defined data based on the \textit{query string} in the URL.
Therefore, we introduce the following optional query string parameters:
\begin{itemize}
	\item \textbf{data:} URL of the data file to load
	\item \textbf{schema:} URL of the schema file to load
	\item \textbf{settings:} URL of the settings file to load
\end{itemize}

Additionally, we introduce a new settings property \textit{toolbarTitle}, which defines the title that is displayed on the \toolname{} toolbar on top of the page.

Figure \ref{fig:custom_configurator} shows the resulting view of \toolname{}, when opening it using the exemplary URL \underline{logende.org/meta-configurator/?data=a\&schema=b\&settings=c}, with \textit{a}, \textit{b}, \textit{c} being URLs to a data, a schema and a settings file.
\toolname{} is opened in the \textit{file editor} mode, with a pre-defined \textit{Self-Driving Vehicle} schema, data file and settings.
The settings file defines the custom title "Autonomous Vehicle Editor", which we can see in the top toolbar in the figure.
Notice that the user will not see the usual schema selection dialog and will not have empty files, but instead start with the pre-defined data.


\begin{figure*}
    \includegraphics[width=\textwidth]{figures/custom_configurator}
    \caption{\toolname{} with a pre-defined \textit{data}, \textit{schema} and \textit{settings} file. The settings file has the property \textit{toolbarTitle} set as "Autonomous Vehicle Editor".}
    \label{fig:custom_configurator}
\end{figure*}


\subsubsection{Advanced support for exporting custom configurators}

With the features described in section \ref{subsubsec:custom_configurator_basic} it is possible for anyone to hand-craft a \toolname{} URL that will load pre-defined data, if the corresponding files are already stored and accessible somewhere.
However, it has the following limitations:
\begin{enumerate}
	\item The user has to store the raw files somewhere, in an accessible manner
	\item The user has to hand-craft the \toolname{} URL with the query string
	\item The resulting URL is very long
\end{enumerate}

To make the process easier for the user, we can add a button in the \toolname{} UI, which will perform the following steps:
\begin{enumerate}
	\item Store user files in backend
	\item Generate URL that refers to the files in the backend
	\item Shorten the URL and store it in the backend too
	\item Return the shortened URL to the user
\end{enumerate}

This way, anyone can generate their own "configurator URL" with the click of a button.

TODO