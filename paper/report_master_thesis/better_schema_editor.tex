Our previously conducted user study suggested that the \textit{schema editor} can be complicated for the user, especially if they have never worked with JSON schema before\cite{metaconfigurator}.


\subsubsection{Making the schema editor more simple}

The reason for the complexity of the schema editor lies in the expressiveness of JSON schema and because \toolname{} supports most of its keywords and features.
This is illustrated by figures \ref{fig:schema_editor_choose_jsonschema}-\ref{fig:schema_editor_choose_type}: when creating a new property for an object, JSON schema allows for the property to be a \textit{boolean schema} (\texttt{true} or \texttt{false}) or a \textit{sub-schema}.
For every \textit{sub-schema}, the \texttt{type} can either be empty, an array of types or a singular type.
All those options that JSON schema offers result in options that the user of the schema editor has to navigate through.

\begin{figure}[!t]
    \centering
    \includegraphics[width=\columnwidth]{figures/schema_editor_choose_jsonschema}
    \caption{Drop-down menu when defining a new property: selection of either a boolean schema or a sub-schema}
    \label{fig:schema_editor_choose_jsonschema}
\end{figure}



\begin{figure}[!t]
    \centering
    \includegraphics[width=\columnwidth]{figures/schema_editor_choose_array_or_single_type}
    \caption{Drop-down menu when defining a new sub-schema: selection of either a single type or a type union}
    \label{fig:schema_editor_choose_array_or_single_type}
\end{figure}



\begin{figure}[!t]
    \centering
    \includegraphics[width=\columnwidth]{figures/schema_editor_choose_type}
    \caption{Drop-down menu when defining a single type}
    \label{fig:schema_editor_choose_type}
\end{figure}


Other schema editors, such as Adamant \cite{todo} and TODO are easier to use, because they are simpler and do not support as many keywords and options as \toolname{}. 
In practice, many JSON schema keywords are used infrequently\cite{baazizi2021empirical} and the full expressiveness of JSON schema is not needed.
To allow for an easier schema editing experience, while maintaining the expressiveness and support for most JSON schema keywords, we introduce JSON meta schema parameters.
The user can decide themselves which advanced JSON meta schema features they want to use, and where they want to work with the simplified meta schema.
We introduce the following boolean parameters:
\begin{itemize}
	\item \textbf{allowBooleanSchema:} Whether a JSON Schema definition can also be just \texttt{true} or \texttt{false}. Having this option enabled will increase the choices that have to be made when defining a sub-schema in the schema editor.
	\item \textbf{allowMultipleTypes:} Whether an object property can be assigned to multiple types (e.g., \texttt{string} and \texttt{number}). Having this option enabled will increase the choices that have to be made when defining the type of a sub-schema in the schema editor, but also allows more flexibility. An alternative to defining multiple types directly is using the \texttt{anyOf} or \texttt{oneOf} keywords.
	\item \textbf{showAdditionalPropertiesButton:} Most schemas allow additional properties (e.g., adding properties to the data that are not defined in the schema). To support this in the schema editor, it would always provide an \textit{Add Property} button to allow adding properties unknown to the schema. In practice, this option is not used much, but it can confuse the user. For example, they might try adding new fields for their schema by using this button, although that does not have any effect on the schema.
	\item \textbf{objectTypesComfort:} This is a comfort feature: the original JSON Meta Schema allows properties of a particular type to have example values, constant values, default values or enum values of different types. For example, a field for numbers could have a string as a default value. This meta schema option forces the same type for all these values. This enables the tool to auto-select the corresponding type in the schema editor, avoiding the need for the user to manually select the types. 
	
	Warning: due to incompatibility, this option will disable schema editor support for defining the items of an array, as well as support for many advanced keywords, such as conditionals and \texttt{not}.
\end{itemize}
The customizability of the schema editor is achieved by a component, which we call \textit{meta schema builder}.
Based on the meta schema parameters, it builds a custom JSON meta schema, which is then used to create the schema editor GUI.
%It starts with our original (advanced) JSON meta schema, and based on the parameters, applies transformations on this meta schema, to simplify it.
We add a button to the schema editor (figure \ref{fig:schema_editor_advanced_mode_button}), which allows the user to toggle between an advanced meta schema (boolean schema and multiple types allowed, no object types comfort feature and showing the button to add additional properties) and a simple one (opposite configuration).
Furthermore, in the settings menu, the user can set the values for the individual parameters.
Figure \ref{fig:schema_editor_simple_choose_type} shows the drop-down menu for defining the type of a property, when using the simplified meta schema.
The type can be set directly.
Compare this with the advanced meta schema, where first the user had to select \texttt{sub-schema}, then \texttt{single type} and only then could select the actual type.

\begin{figure}[!t]
    \centering
    \includegraphics[width=0.5\columnwidth]{figures/schema_editor_advanced_mode_button}
    \caption{Button to enable the advanced schema editor, by using a more advanced meta schema}
    \label{fig:schema_editor_advanced_mode_button}
\end{figure}

\begin{figure}[!t]
    \centering
    \includegraphics[width=\columnwidth]{figures/schema_editor_simple_choose_type}
    \caption{Drop-down menu when defining the type of a new property}
    \label{fig:schema_editor_simple_choose_type}
\end{figure}



\subsubsection{Providing a (functional) preview of the resulting GUI}

The file editor GUI, which is created from the user schema, helps the user understand their schema.
It shows the properties defined in the schema, the titles and descriptions in the schema, whether a property is \texttt{required} or \texttt{deprecated}, and more.
To access the file editor, the user has to navigate from the schema editor view to the file editor view.
They cannot view both the schema editor and the resulting GUI simultaneously.
However, the improved architecture, as described in section \ref{subsubsec:dynamic_panels}, does make different panel arrangements possible.
We make use of that architecture and add a button to the schema editor (figure \ref{fig:schema_editor_show_preview}), which allows the user to toggle between showing the resulting file editor GUI in a third panel and hiding it. 


\begin{figure}[!t]
    \centering
    \includegraphics[width=0.5\columnwidth]{figures/schema_editor_show_preview}
    \caption{Button to show the file editor GUI in a third panel}
    \label{fig:schema_editor_show_preview}
\end{figure}
